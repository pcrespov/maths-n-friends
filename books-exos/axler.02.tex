% -*- Mode: LaTeX; tab-width: 4 -*-
%
% Work on Sheldon Axler 'Linear Algebra Done Right'.
% Pedro Crespo Valero - 2015, 2021

%  Chapter 2

%%%%%%%%%%%%%%%%%%%%%%%%%%%%%%%%%%%%%%%%%%%%%%%%%%%%%%%%%%%%%%%

\section{Finite-Dimensional Vector Spaces}
\subsection*{Summary}
\begin{itemize}
\item \textbf{linear combination} of a list $(v_1, \cdots, v_m)$ of vectors in $V$ is a vector $u = \sum a_i v_i$ with $a_i\in \mathbf{F}$
\item \textbf{span} of $(v_1, \cdots, v_m)$ is the set $\mathrm{span}(v_1, \cdots, v_m) = \{ \sum a_i v_i \mid a_i\in\mathbf{F}\}$
\begin{itemize}
  \item is the set of all linear combinations
  \item is a subspace of $V$
  \item for an empty list: $\mathrm{span}() = \{0\}$
  \item if $\mathrm{span}(v_1, \cdots, v_m) = V$ (i.e. $(v_1, \cdots, v_m)$ spans $V$ ) then $\mathrm{span}(v_1, \cdots, v_m)$ is \emph{the smallest subspace} in $V$
  \item a vector space $V$ is \textbf{finite dimensional} if exists a finite list $(v_1, \cdots, v_m)$ that spans $V$
\end{itemize}
\item A polynomial $p(z)$ 
\begin{itemize}
  \item has \textbf{degree} $m$ if $p(z)=a_0+a_1z+\cdots+a_mz^m$ with $a_i\in \mathbf{F}$ and $a_m\neq 0$
  \item $p(z)=0$ has degree $-\infty$ ???
  \item $\mathcal{P}_m(\mathbf{F})$ set of all polynomials with degree $\leq m$
  \item $\mathcal{P}_m(\mathbf{F})$ is a finite dimensional subspace of $\mathcal{P}(\mathbf{F})$
\end{itemize}
\item none finite dimensional spaces are \textbf{infinite dimensional} ex. $\mathcal{P}(\mathbf{F})$ or $\mathbf{F}^\infty$
\item A list of vectors $(v_1,\cdots,v_n)$ is \textbf{linearly independent} in $V$ $\iff 0 = \sum a_i v_i$
and the only choice is $a_1=\cdots=a_n=0$. Otherwise is \textbf{linearly dependent}
\begin{itemize}
  \item a list $(v,)$ of length 1 is l.i. iff $v\neq 0$
  \item a list of length 2 is l.i. neither vector is scalar multiple of the other
  \item a list of length 3 or more is l.d. even thought no vector in the list is scalar multiple of another!
  \item removing vectors from a l.i. list remains l.i.
  \item we declare empty list $()$ as l.i.
\end{itemize}

\item[L4]\emph{Linear Dependence Lemma}: If $(v_1, \cdots, v_m)$ l.d. in $V$ and $v_1\neq 0$ then $\exists j\in 2,\cdots m$ such that
\begin{itemize}
  \item $v_j \in \mathrm{span}(v_1,\cdots, v_{j-1})$
  \item  $\mathrm{span}(v_1,\cdots, v_{j-1}, v_{j+1}, \cdots, v_m) = \mathrm{span}(v_1,\cdots, v_m)$
\end{itemize}
\item[T6:\label{it:T2_6}] In a \emph{finite} dimensional vector space, the \emph{length}(of every list of indepedent vectors) $\leq$ of the \emph{length}(of every spanning list of vectors)
\item[P7:] Every subspace $S$ of a \emph{finite}-dimensional vector space $V$ (i.e. $S\subset V$) is also \emph{finite}-dimensional
%

\item A \textbf{basis} of a vector space $V$ is a list of vectors in $V$ that is \emph{linearly indepedent} and \emph{spans} $V$.
\begin{itemize}
\item $(e_1, \cdots, e_n)$ where $e_i=(0,\cdots, \underbrace{1}_{i},\cdots,0)\in\mathbf{F}^n$ is the \textbf{standard basis} of $\mathbf{F}^n$. 
\item By \tref{it:T2_6}, a basis of $V$ is the smallest list of l.i. vectors that spans $V$
\end{itemize}
\item[P8:] A list $(v_1,\cdots,v_n)$ of vectors in $V$ is a basis of $V$ $\iff$ $\forall v\in V$ can be written uniquely as a linear combination of these vectors $v=\sum a_i v_i$.
\item[T10/12:] Every \emph{spanning/linearly independent}  list of vectors in a finite-dimensional vector space can be \emph{reduced}/\emph{extended} to a basis of the vector space.
\item[C11:] Every finite dimensional vector space \emph{has} a basis.
\item[P13:] $V$ finite dimensional and $U$ subspace of $V$ $\implies$ there is a subspace $W$ of $V$ such that $V=U\oplus W$.
%
\item[T14:] Any two bases of a finite-dimensional vector space have the same length
\item The \textbf{dimension} of a finite-dimensional vector space $\dimension{V}$ is the \emph{length} of any basis of the vector space
\begin{itemize}
  \item $\dimension{\mathbf{F}^n} = n$
  \item $\mathcal{P}(\mathrm{F}_m) = m+1$
\end{itemize}
\item[P15:] if $U$  subspace of finite-dimensional $V$ $\implies$ $\dimension{U}\leq\dimension{V}$.
%
\item[P16/17] If a list has the right lenght, only need to verify one of the props to be base: either span or l.i.:
\begin{itemize}
  \item[P16:\label{it:T2_16}]Every spanning list of vectors in $V$ reduced to a length $\dimension{V}$ is a basis of $V$ 
  \item[P17:\label{it:T2_17}]Every linearly-independent list of vectors in $V$ extended to a lenght $\dimension{V}$ is a basis of $V$
\end{itemize}
%
\item[T18:\label{it:T2_18}] If $U_1,U_2$ subspaces of a finite-dimensional vector space then 
\begin{align*}
\dimension(U_1+U_2)=\dimension(U_1)+\dimension(U_2) - \dimension(U_1\cap U_2)
\end{align*}
\item[P19:\label{it:T2_19}] $V$ finite dimensional vector space and $U_1,\cdots,U_m$ subspaces of $V$ such that 
\begin{itemize}
  \item $V=\sum U_i$
  \item $\dimension{V}=\sum\dimension{U_i}$
\end{itemize}
 then $V=U_1 \oplus \cdots \oplus U_n = \bigoplus U_i$
\end{itemize}



%------------------------------------------------------------------------ 
\subsection*{Exercises}
%------------------------------------------------------------------------ 
\exo{}
\begin{align*}
a_1(v_1 - v_2) + a_2 (v_2 - v_3) + \cdots + a_{n-1} (v_{n-1} - v_n) + v_n = \\
a_1 v_1 + (a_1 - a_2) v_2 + \cdots + (1 - a_{n-1}) v_n 
\end{align*}
since the latter spans V, then $0 \neq a_1\neq a_2 \cdots$ 
\qed

%------------------------------------------------------------------------ 
\exo{} If $(v_1,\cdots,v_n)$ is l.i. in $V$ then $\sum a_i v_i = 0$ and $a_i = 0\; \forall i$, so any l.c. of the new list 
\begin{align*}
  c_1 (v_1 - v_2) + c_2 (v_2 - v_3) + \cdots + c_n v_n &= 0\\
  c_1 v_1 + (c_2 - c_1) v_2 + \cdots (c_n - c_{n-1})v_n &= 0
\end{align*}
so $c_1=0$, $c_2 = c_1$, ...  so it is l.i. as well \qed
%------------------------------------------------------------------------ 
\exo{} Let $\sum a_i (v_i + w) = \sum a_i v_i + (\sum a_i) w = 0$ and since some $a_j\neq 0$ then $\sum a_i\neq 0$ 
which leads to $w = \sum \frac{-a_i}{\sum a_i} v_i$ \qed 

%------------------------------------------------------------------------ 
\exo{} No, since it is not close in addition. E.g. the sum of two of its elements $ (1 + z^m) + (- z^m) = 1  $ 
results in a polynomial with smaller degree. In contrast note that $\mathcal{P}(\mathrm{F}_m)$ (with degree $\leq m$ ) 
\emph{is} a subspace. 

%------------------------------------------------------------------------ 
\exo{}
\hilight{TODO}

%------------------------------------------------------------------------ 
\exo{}
\hilight{TODO}

%------------------------------------------------------------------------ 
\exo{}
\hilight{TODO}

%------------------------------------------------------------------------ 
\exo{} $\forall u\in U$ is given by $u = (3\alpha , \alpha, 7\beta, \beta, \lambda)$ with $\alpha,\beta,\lambda \in \mathbf{R}$, which can be written 
as $ u = \alpha \underbrace{(3 , 1, 0, 0, 0)}_{u_1} + \beta\underbrace{(0 , 0, 7, 1, 0)}_{u_2} + \lambda \underbrace{(0 , 0, 0, 0, 1)}_{u_3} $ therefore 
 $(u_1,u_2,u_3)$ span $U$. In addition, they are l.i. in $\mathbf{R}^5$ and therefore a base of $U$.

%------------------------------------------------------------------------ 
\exo{} Yes. The list ($1$, $z$, $z^3+z^2$, $z^3$) is a base of $\mathcal{P}_3(\mathbf{F})$ and none of the polynomials (vectors) have degree 2. Note that $z^2 = \mathbf{p}_2 - \mathbf{p}_3$.

%------------------------------------------------------------------------ 
\exo{} $V$ with $\mathrm{dim}V = n$ then there is a base $(v_1, \cdots, v_n)$ which by definition $u = \sum a_i v_i$ and $a_i$ are unique. 
Take each of $v_i$ to span a one dimensional subspace $U_i$, we clearly have that $ V = \bigoplus U_i $


%------------------------------------------------------------------------ 
\exo{}  Let $B_1$ bases of subspace $U\subseteq V $, then the list $B_1$ is also l.i. in $V$ with length $\dimension{U} = \dimension{V}$ 
therefore, by \pref{it:T2_17}, is also a bases of $V$ which means that both are the same set $U=V$. 


%------------------------------------------------------------------------ 
\exo{} $a_i=0$ is not the only solution $\forall z\in \mathbf{F}$ for $\sum a_i \mathbf{p}_i(z) = 0$ since at $z=2$
all polynomials have a root, i.e. $\mathbf{p}_i(2)=0$.  Therefore it cannot be linearly indepedent. 
\hilight{???}

%------------------------------------------------------------------------ 
\exo{} By \tref{it:T2_18}, 
\begin{align*}
  \underbrace{\dimension{R^8}}_8 = \dimension{U+W} = \underbrace{\dimension{U}}_3 + \underbrace{\dimension{W}}_5 - \dimension{U \cap W} \implies \dimension{U \cap W} = 0
\end{align*}
which leads to the only subspace dimensionless, i.e. $U \cap W = \{0\}$
%------------------------------------------------------------------------ 
\exo{} Following previous reasoning leads to $ \dimension{U \cap W} = 1 \implies U\cap W \neq \{0\}$

%------------------------------------------------------------------------ 
\exo{} No. Take $u_1,u_2,u_3$ three vectors in $\mathbf{R}^2$ and bases spanning $U_1, U_2, U_3$ resp., then 
\begin{enumerate}
  \item $\dimension{(U_1+U_2+U_3)} = \dimension{\mathbf{R}^2} = 2$
  \item $\dimension{U_i} = 1$
  \item $U_i\cap U_j = \{0\}$ for $i\neq j$ 
\end{enumerate} 
but then $\dimension{(U_1+U_2+U_3)} \neq \dimension{U_1} + \dimension{U_2} + \dimension{U_3} - 0 - ... + 0$.



%------------------------------------------------------------------------ 
\exo{}  Using \tref{it:T2_18} we have that $\dimension{  \sum U_i  } = \sum \dimension{U_i} - \sum \underbrace{\dimension{ (U_i \cap \sum_{j>i} U_j) }}_{\ge 0}  $
$\implies  \dimension{  \sum U_i  } \leq \sum \dimension{U_i}  $

%------------------------------------------------------------------------ 

\exo{} Following from previous result, $U_i \cap \bigoplus\limits_{j>i} U_j = \{0\}$ because the bases of each $U_i$ are l.i. .

