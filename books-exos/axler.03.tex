% -*- Mode: LaTeX; tab-width: 4 -*-
%
% Work on Sheldon Axler 'Linear Algebra Done Right'.
% Pedro Crespo Valero - 2015, 2021

%  Chapter 3

%%%%%%%%%%%%%%%%%%%%%%%%%%%%%%%%%%%%%%%%%%%%%%%%%%%%%%%%%%%%%%%

\section{Linear Maps}
\subsection*{Summary}

\begin{description}
  \item[Linear Map] $T:V\to W$ with the following properties
  \begin{itemize}
  \item additivity $T(u+v) = Tu +Tv \in W$ for all $u,v\in V$
  \item homogeneity $T(av) = aTv \in W$ for all $a\in \mathbf{F}$
  \end{itemize}
  %
  \item[Set of Linear Maps\label{itm:D3_set_linear_maps}] $T\in \mathcal{L}(V,W)$
  \begin{itemize}
  \item $0$ map: $0v = 0$ and the first $0\in \mathcal{L}(V,W)$
  \item $I$ map: $Iv=v$ and $I\in \mathcal{L}(V,V)$
  \item differentiation map : $Tp = p'$ and $T\in \mathcal{L}(\mathcal{P}(\mathbf{R}),\mathcal{P}(\mathbf{R}))$
  \item integration map: $Tp = \int\limits_0^1p(x)dx$ and $T\in \mathcal{L}(\mathcal{P}(\mathbf{R}),\mathbf{R})$
  \item multiplication by $x^2$ map: $Tp=x^2p(x)$ with $T\in \mathcal{L}(\mathcal{P}(\mathbf{R}),\mathcal{P}(\mathbf{R}))$
  \item backwards shift map: $T x = T(x_1,x_2, \cdots) = (x_2, x_3, \cdots)$ with $T\in \mathcal{L}(\mathbf{F}^\infty, \mathbf{F}^\infty)$
  \item map $T:\mathbf{F}^n \to \mathbf{F}^m$: 
  \begin{align*}
  T x = T(x_1,\cdots,x_n) =  (T_1x,  \cdots, T_m x) = (\sum a_{1i}x_i,  \cdots, \sum a_{mi}x_i) = y
  \end{align*} with $a_{ij}\in\mathbf{F}$.
  
  \item A linear map can be constructed as follows: Take a basis $(v_1,\cdots,v_n)$ of $V$ and any choice $w_1,\cdots,w_n$  of $W$ and then define $T$ as
  \begin{align*}
  Tv = T(\sum a_i v_i) \eqdef \sum a_i w_i = w
  \end{align*}
  for arbitrary $a_i\in\mathbf{F}$. Then the equation above:
  \begin{itemize}
    \item is a function from $v\in V$ to $w\in W$
    \item is linear map since $T(\alpha v+ \beta v') = T(\alpha \sum a_i v_i + \beta \sum  b_i v_i) = T(\sum (\alpha a_i +  \beta b_i) v_i) \eqdef \sum (\alpha a_i +  \beta b_i) w_i = \alpha \sum a_i w_i + \beta \sum b_i w_i \eqdefi \alpha Tv + \beta Tv' $  
    \item and the transformation on the basis vectors is $Tv_i=w_i$, by simply taking $a_j=0$ for $\forall j\ne j$ except for $a_i=1$.
  \end{itemize}
  \end{itemize}
  
  \item[Vector Space] $\mathcal{L}(V,W)$  by \emph{defining} the following operations
  \begin{itemize}
  \item $(S+T)v = Sv + Tv \in W$ 
  \item $(aT)v = a(Tv) \in W$
  \end{itemize}
  $\forall S,V \in \mathcal{L}(V,W)$ and $a\in\mathbf{F}$ it satisfies
  \begin{itemize}
  \item closed under sum: $S+V \in \mathcal{L}(V,W)$
  \item closed under scalar multiplication: $aT \in \mathcal{L}(V,W)$
  \item additive identity is the zero map: $T+0=T$
  \end{itemize}
  %
  \item[Multiplication] (=composition) of linear maps: Let $U \xrightarrow{T} V \xrightarrow{S} W$ then we define
  \begin{align*}
  (ST)v = S(Tv)
  \end{align*}
  has the properties of
  \begin{itemize}
  \item associativity: $(T_1T_2)T_3 = T_1(T_2T_3)$.
  \item identity: $TI = T = IT$.
  \item distributive: $(S_1+S_2)T = S_1T+S_2T$ and $S(T_1+T_2) = ST_1 +ST_2$ for $U\xrightarrow{S,S_1,S_1} V \xrightarrow{T,T_1,T_2} W$.
  \item \emph{not} conmutative: in general $ST \neq TS $
  \end{itemize}
  %
  \item[null space\label{itm:D3_nullspace}] of $T$: $\nullspace{T} = \{v\in V \mid Tv=0\}$
  \item[range] (or image) of $T$: $\rangespace{T} = \{Tv \mid v\in V\}$
  \item[P1\label{itm:P3_1}]If $T\in\mathcal{L}(V,W)$ then $\nullspace{T}$ is a \emph{subspace} of $V$
  \item[P3]If $T\in\mathcal{L}(V,W)$ then $\rangespace{T}$ is a \emph{subspace} of $W$
  \item[P2\label{itm:P3_2}]$T\in\mathcal{L}(V,W)$ \textbf{injective} $\iff \nullspace{T}=\{0\}$
  \begin{itemize}
  \item $T:V\to W$ is injective if $\forall u,v \in V$ and $Tu=Tv$ then $u=v$
  \end{itemize}
  \item[surjective] $T\in\mathcal{L}(V,W)$ is surjective $\iff  \rangespace{T} = W$ (by definition)
  \item[T4] If $V$ is finite dimensional and $T\in\mathcal{L}(V,W)$, then
  \begin{itemize}
  \item $\rangespace{T}$ is finite dimensional in $W$
  \item the dimension of $V$ can be expressed as
  \begin{align*}
  \Aboxed{\dim{V} = \dimension{\nullspace{T}} + \dimension{\rangespace{T}}}
  \end{align*}
  \end{itemize}
  %
  \item $V,W$ finite dimensional and $T\in\mathcal{L}(V,W)$ (dimension as "size")
  \begin{itemize}
  \item[C5] $\dimension{V} > \dimension{W} \implies \nexists\,T$ injective
  \item[C6] $\dimension{V} < \dimension{W} \implies \nexists\,T$ surjective
  \end{itemize}
  %
  \item [Theory of linear equations] C5 and C6 have consequences. Let $T:\mathbf{F}^n\to\mathbf{F}^m$ be defined as 
  \begin{align*}
  Tx &= y \\
  T(\sum^n x_i e_i) &= \sum^m y_j e'_j \\
  T(x_1,\cdots,x_n) &= (y_1,\cdots,y_m) \eqdef (\sum^n a_{1k}x_k, \cdots,\sum^n a_{mk}x_k)
  \end{align*}
  then the system of \emph{homogeneous} equations can be written as $Tx=0$ and
  \begin{itemize}
  \item Solutions are $x\in\nullspace{T}$.
  \item $\dimension{\nullspace{T}}>0$ iff $T$ not injective.
  \item  $T$ not injective if $\dimension{V} > \dimension{W}$ or $n>m$ (i.e. more unknown than equations).
  \end{itemize}
  The system of \emph{inhomogeneous} equations can be written as $Tx=c$
  \begin{itemize}
  \item $\forall c\in\rangespace{T}$ exists a solution in $x$
  \item but every solution $c\in \mathbf{F}^m$, i.e. is $\rangespace{T}=\mathbf{F}^m$?
  \item only if $T$ surjective.
  \item $T$ not surjective if $\dimension{V} < \dimension{W}$ or $n<m$ (i.e. smaller number of unknowns than equations).
  \item An inhomogenous system of linear equations with more equations than variables has no solution for some choice of the constant terms $c$.
  \end{itemize}
  %
  \item[Matrix Of a Linear Map\label{itm:D6_matrix}]  $T\in\mathcal{L}(V,W)$ is given by \begin{align*}
  \mathcal{M}(T, (v_1,\cdots, v_n), (w_1, \cdots, w_m)) &=\begin{bmatrix}
  Tv_1 &| \cdots |& Tv_k & |\cdots |& Tv_n 
  \end{bmatrix} 
  \end{align*} by expanding every term of a base of $V$ as linear combinations of a base in $W$. Let us expand both sides of $w = Tv$:
  \begin{align*}
  \sum_i b_i w_i &= T\left(\sum_j c_jv_j\right) = \sum_j c_j T v_j = \sum_j c_j \left(\sum_i a_{ij} w_i\right) = \sum_i \left( \sum_j  a_{ij} c_j\right) w_i 
  \end{align*}
  Therefore the coefficients of every expansion are related by $b_i = \sum_j  a_{ij} c_j$, which can be expressed as:
  \begin{align*}
  \begin{bmatrix}
  b_1 \\
  \vdots \\
  b_i \\
  \vdots \\
  b_m
  \end{bmatrix}
  &= \begin{bmatrix}
  a_{11} &       &a_{1j} &        & a_{1n} \\
         &       &\vdots &        &        \\
  \vdots &\cdots &a_{ij} & \cdots &\vdots  \\
         &       &\vdots &        &        \\
  a_{m1} &       &a_{mj} &        &a_{mn}  \\
  \end{bmatrix}
  \begin{bmatrix}
  c_1 \\
  \vdots \\
  c_j \\
  \vdots \\
  c_n
  \end{bmatrix}
  \end{align*} where
  \begin{itemize}
  \item $c_i$ are the coefficients of the linear combination of $v$ on a base $\{v_j\}$ in $V$.
  \item $b_i$ are the coefficients of the linear combination of $w$ on a base $\{w_i\}$ in $W$.
  \item $a_{ij}$ coefficients of the linear combination of every vector in the base $v_j$ on the base $\{w_i\}$ in $W$.
  \end{itemize}
  Properties:
  \begin{itemize}
  \item $\mathcal{M}(T+S) = \mathcal{M}(T) + \mathcal{M}(S)$
  \item $\mathcal{M}(cT) = c\mathcal{M}(T)$
  \item $\mathcal{M}(PT) = \mathcal{M}(P)\mathcal{M}(T)$
  \end{itemize}
  whenever $c\in\mathbf{F}$, $T,S\in\mathcal{L}(U,V)$ and  $P\in\mathcal{M}(V,W)$
  %
  \item $\mathrm{Mat}(m,n,\mathbf{F})$ is the set of all $m\times n$ matrices with entries in $\mathbf{F}$. It is a vector space over $\mathbf{F}$
  \item \emph{Matrix of a vector} $v\in V$ is a $n\times 1$ matrix.
  \item[P14] $T\in\mathcal{L}(V,W)$ and $(v_1,\cdots,v_n)$ a base of $V$ and $(w_1,\cdots,w_m)$ a base of $W$. Then
  \begin{align*}
  \mathcal{M}(Tv) =\mathcal{M}(T)\mathcal{M}(v) 
  \end{align*}
  %
  \item[Invertibility\label{itm:D3_invertible_operator}] $T\in\mathcal{L}(V,W)$ \textbf{invertible} if $\exists S\in\mathcal{L}(W,V)$ and $ST=I_V$ (identity map on $V$) and $TS=I_U$ (identity map on $W$) 
  \begin{align*}
  I \circlearrowright V \xrightarrow{T}\xleftarrow{S} W \circlearrowleft I
  \end{align*}
  \item If $T$ invertible then it has a unique inverse denoted $T^{-1}$ 
  \item[P17] A linear map is invertible $\iff$ it is injective and surjective.
  
  \item[Isomorphic] Two vectors spaces are called isomorphic if there is an invertible linear map from one vector space onto the other one. As abstract vectors spaces, two isomorphic spaces have the same properties.
  
  \item[T18] Two finite dimensional vector spaces are isomorphic $\iff$ they have the same dimension
  
  \item[P19]
  
  \item[P20\label{itm:P3_20}] If $V$ and $W$ are finite dimensional vector spaces, then
  \begin{itemize}
  \item $\mathcal{L}(V,W)$ is a finite dimensional vector space
  \item $\dim{\mathcal{L}(V,W)} = \dim{V} \times \dim{W}$
  \end{itemize}
  
  \item[T21\label{itm:T3_21}] $V$ finite dimensional and $T\in\mathcal{L}(V)$ then
  \begin{align*}
  T \text{ invertible} \iff T \text{ injective} \iff T \text{ surjective} 
  \end{align*}
  
  
\end{description}
  
\hilight{TODO}
  
%------------------------------------------------------------------------ 
\subsection*{Exercises}
\setcounter{paragraph}{0}

\exo{} With $\dim{V}=1$ and $T\in\mathcal{L}(V,V)$ then let $\{v\}$ be a base of $V$ s.t. $\forall u\in V \implies u=av$. 
Therefore the linear map on $u$ is by definition
\begin{align*}
  Tu = T(av) = a Tv = abv = cv
\end{align*}
since $Tv \in V$ and can be expanded with the base $v$ as well. So under these conditions, 
\begin{align*}
  Tu = c v
\end{align*}
for $c\in\mathbf{F}$ and $u,v\in V$.

---

If $T \in \mathcal{L}(V,V)$ then by T3.4
\begin{align*}
\dim{V} = \underbrace{\dim{\nullspace{T}}}_{\leq 1} + \underbrace{\dim{\rangespace{T}}}_{\leq 1} = 1
\end{align*}
since $\nullspace{T}$ and $\rangespace{T}$ are subspaces of $V$ (source and destination domains are $V$) its dimensions are at most 1 (3.3 and 2.15). Two cases are possible
\begin{itemize}
\item $\dim{\rangespace{T}}=0$ and $\dim{\nullspace{T}}=1$ implies that $\rangespace{T}=\{0\}$ (subspace of dimension zero) and $\nullspace{T}=V$ (subspace with same dimension of containing vector space 2.12).
\item $\dim{\rangespace{T}}=1$ and $\dim{\nullspace{T}}=0$ means that $\rangespace{T}=V$ and $\nullspace{T}=\{0\}$ 
\end{itemize}

$\rangespace{T}=V$ then $Tv = w$ can be expanded in a bases of $V$. Since $\dim{V}=1$ any vector $v\in V$ can be a base. Therefore $Tv = a v$ where $a\in \mathbf{F}$. Notice that if $a=0$, the first case is also included.

%------------------------------------------------------------------------ 
\exo{}
\begin{align*}
  f(v) = \sqrt{v_1 v_2}
\end{align*} 
where $v=v_1e_1+v_2e_2 \equiv (v_1,v_2) \in \mathbf{R}^2$. Then if $f(av) = \sqrt{av_1 av_2} = a\sqrt{v_1v_2} = af(v)$

\hilight{a negative!}

%------------------------------------------------------------------------ 
\exo{} By \pref{itm:P2_13} P2.13 we can extend subspace $U$ with another subspace $U'\subset V$ such that $U\oplus U' = V \implies v = u+u'$ 
is a unique decomposition and if we define 
\begin{align*}
Tv = Tu+Tu' = Su + S'u'
\end{align*}  
with $S\in\mathcal{L}(U,W)$, $S'\in\mathcal{L}(U',W)$ and $T\in\mathcal{L}(U\oplus U',W)$. 
And if $v=u \implies Tv = Su$.

%------------------------------------------------------------------------ 
\exo{} By \pref{itm:P2_13} we can extend $\nullspace{T}$ subspace of $V$ (\pref{itm:P3_1}) to $V$ as $V = \nullspace{T} \oplus W$. Hence
\begin{align*}
\dimension{V} \stackrel{\text{T2.18}+\text{P1.6}}{=} \dimension{\nullspace{T}} + \dimension{W} \stackrel{\text{T3.4}}{=} \dimension{\nullspace{T}} + \dimension{\rangespace{T}}
\end{align*} 
but the $\rangespace{T}\subseteq \mathbf{F}$ (P3.3) since $T\in\mathcal{L}(V,\mathbf{F})$. Finally $\dimension{\rangespace{T}}\leq\dimension{F}=1$ (P2.15) which means, considering the equation above, that $\dimension{W}\leq 1$ and hence can be spanned by a single-vector base $W=\{au \mid a\in\mathrm{F}\}$

%------------------------------------------------------------------------ 
\exo{} $(v_1,\cdots,v_n)$ l.i. in $V$ implies $0=\sum a_iv_i$ and $a_i=0\,\forall i$. $T$ injective $\stackrel{P3.2}{\iff} \nullspace{T}=\{0\}$ then $Tv_i\neq 0$ and $Tv_i\neq Tv_j$. Hence 
\begin{align*}
T0 = 0 = a_1 Tv_1 + \cdots + a_n Tv_n
\end{align*}
where all $a_i=0$, which means that $(Tv_1,\cdots,Tv_n)$ are l.i. 
%------------------------------------------------------------------------ 
\exo{} Let us assume that $S_1S_2\cdots S_n(x) = S_1S_2\cdots S_n(y)$. 
Rewriting $S_1(S_2( \cdots S_n(\alpha) )) = S_1 \alpha_1$ we can shorten it as $S_1 x_1 = S_1 y_1$. 
Since $S_1$ is injective this implies that $x_1 = y_1$ or equivalently $S_2( S_3 \cdots S_n(x)) = S_2( S_3 \cdots S_n(y))$. 
Repeating the same reasoning $n$ times, we reach that $x=y$ and therefore can conclude that the product 
$S_1S_2\cdots S_n$ is injective. 

%------------------------------------------------------------------------ 
\exo{} $(v_1,\cdots,v_n)$ spans $V$ and $T\in\mathcal{L}(V,W)$ is surjective, and 
so the map $Tv$ reaches any vector in $W$. Then $\forall w\in W$ 
\begin{align*}
  w = Tv = \sum^n a_i Tv_i
\end{align*}
or in other words, the list $(Tv_1,\cdots, Tv_n)$ spans $W$. Definitively some of the $Tvi=0$ which 
suggests that $\dim{V}\geq \dim{W}$ as can also be deduced from C3.6. 


%------------------------------------------------------------------------ 
\exo{} By P3.1 $\nullspace{T}$ is a subspace of $V$ and since $U \cap \nullspace{T} = \{0\}$ by P1.6 
we can decompose $V$ as $V = \nullspace{T} \oplus U$, then $\forall v\in V$ can be uniquely expressed as 
$v=n+u$ which maps as $Tv=0+Tu \in W$, that is, $U \xrightarrow{T} \rangespace{T} \subseteq W$.

%------------------------------------------------------------------------ 
\exo{} Let  $T\in\mathcal{L}(\mathbf{F}^4, \mathbf{F}^2)$. $\nullspace{T}$ can be span with two linearly independent
vectors (i.e. a base) as
\begin{align*}
  x &= 5\alpha e_1 + \alpha e_2 + 7\beta e_3+ \beta e_4 \\
  x &= \alpha (5e_1 + e_2) + \beta (7e_3 + e_4) \\
  x &= \alpha (5,1,0,0) + \beta (0,0,7,1)
\end{align*}
, therefore $\dimension{\nullspace T} = 2$. 
Using the identity from T4: $\dimension{\mathbf{F}^4} = 4 = 2 + \dimension{\rangespace{T}} \implies \dimension{\rangespace{T}} = \dimension{\mathbf{F}^2} = 2$ 
therefore $T$ is surjective. 
%------------------------------------------------------------------------ 
\exo{} This subspace can be span with two l.i. vectors $(v_1, v_2)$
\begin{align*}
  x &= 3\alpha e_1 + \alpha e_2 + \beta e_3+ \beta e_4 + \beta e_5\\
  x &= \alpha \underbrace{(3e_1 + e_2)}_{v_1} + \beta \underbrace{(e_3 + e_4 + e_5)}_{v_2} \\
  x &= \alpha (3,1,0,0,0) + \beta (0, 0, 1,1,1)
\end{align*}
so it has dimension two. Now if this would be a nullspace, using T4, would mean that $\dimension{\rangespace{T}} = 3$
but  $\rangespace{T} \subseteq \mathbf{F}^2$ which is a contradiction.

%------------------------------------------------------------------------ 
\exo{} $\dimension{T} = \dimension{\nullspace{T}} + \dimension{\rangespace{T}} < \infty$
%------------------------------------------------------------------------ 
\exo{} $\exists T \in\mathcal{L}(V,W)$ surjective $ \iff \dimension{W}\leq \dimension{V}$
\begin{itemize}
  \item[$\Rightarrow$] If $\exists T$ surjective then by definition $\rangespace{T} = W$ and using T3.4 
  $\dimension{V} = \dimension{\nullspace{T}} + \dimension{W} \implies \dimension{V}\geq \dimension{W}$
  \item[$\Leftarrow$] if $n=\dimension{V}\geq \dimension{W}=m$ we can construct a linear map using the method above and
  choosing basis in both vector spaces. This is, let's define a linear map as $Tv = \sum^n a_i w_i$ with a base 
  $(v_1, \cdots, v_n)$ of $V$ and choosing a base plus some l.d. vectors $(w_1, \cdots, w_m, w_{m+1}, \cdots, w_n)$ . 
  Since the latter list spans $W$, $Tv$ will do as well, meaning that $\rangespace{T} = W$.
  \hilight{??}
\end{itemize}

%------------------------------------------------------------------------ 
\exo{} $T\in\mathcal{L}(V,W)$ and $\nullspace{T}=U \subset V \iff \dimension{V} - \dimension{W} \leq \dimension{U}$
\begin{itemize}
\item[$\Rightarrow$] $\dimension{V}=\dimension{\nullspace{T}} + \dimension{\rangespace{T}}\leq \dimension{\nullspace{T}} + \dimension{W}$ provided that $\rangespace{T}\subseteq W$. Hence $\dimension{U} \geq \dimension{V} - \dimension{W}$. 
\item[$\Leftarrow$] If $\dimension{V} - \dimension{W} \geq 0 \iff \dimension{V} \geq \dimension{W}$ implies (by 3.12) that $\exists T$ surjective
i.e. $\rangespace{T} = W$ and therefore $\dimension{V} - \dimension{W} = \dimension{\nullspace{T}}$ that suggests that $U=\nullspace{T}$.
\end{itemize}

%------------------------------------------------------------------------ 
\exo{}

\hilight{TODO}

