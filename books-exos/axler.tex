% -*- Mode: LaTeX; tab-width: 4 -*-
%
% Work on Sheldon Axler 'Linear Algebra Done Right'.
% Pedro Crespo Valero - 2015, 2021

\documentclass[11pt,notitlepage,oneside]{article}
\author{Pedro Crespo-Valero}

\usepackage[margin=1.5cm]{geometry} % to tell the pdf about our paper size
\usepackage[pdfpagelabels,pdfpagemode=None,backref=page,pagebackref=true,
colorlinks=true,bookmarks=true,bookmarksnumbered=true]{hyperref}
\usepackage{nameref}
\usepackage[latin1]{inputenc} % accents, REQUIRED for non english languages
\usepackage[english]{babel}
\usepackage{graphicx} % anything beyond pure text requires this
\usepackage[centertags,sumlimits,nointlimits,reqno]{amsmath}
\usepackage{amssymb}
\usepackage{amsbsy}
\usepackage{amsxtra}
\usepackage{amsthm}
\usepackage{mathtools}
\usepackage[nottoc,notbib]{tocbibind} % to put the bibliography in the toc
\usepackage[square,comma,numbers,sort&compress]{natbib} % many options in bibl.
\usepackage{tikz}
\usetikzlibrary{calc}
\usetikzlibrary{shapes,backgrounds}
%\usepackage{color}
%\usepackage{enumitem}
\usepackage{pifont}% http://ctan.org/pkg/pifont
\usepackage[printonlyused,withpage]{acronym}

%% NEW COMMANDS ------
\newcommand{\cmark}{\ding{51}}%
\newcommand{\xmark}{\ding{55}}%
\newcommand{\hilight}[1]{\colorbox{yellow}{#1}}
\newcommand{\pfrac}[2]{\frac{\partial #1}{\partial #2}}
\newcommand{\diff}{\mathrm{d}}
\newcommand{\firstsaid}[1]{\textbf{#1}}
\newcommand{\relphantom}[1]{\mathrel{\phantom{#1}}}
\newcommand{\wedgeseq}{\wedge\ldots\wedge}
\newcommand{\hodge}{\star}
\newcommand{\laplacian}{\Delta}
\DeclareMathOperator{\rot}{curl}
\DeclareMathOperator{\divg}{div}
\DeclareMathOperator{\dimension}{dim}
\DeclareMathOperator{\grad}{grad}
\DeclareMathOperator{\trace}{trace}
\DeclareMathOperator{\conv}{conv}

\renewcommand{\theparagraph}{\arabic{section}.\arabic{paragraph}}
\newcommand{\exref}[1]{Ex.\ref{#1}}
\newcommand{\figref}[1]{Fig.\ref{#1}}
\newcommand{\pref}[1]{Pr.\ref{#1}}
\newcommand{\tref}[1]{Th.\ref{#1}}

\newcommand{\tbc}{\ensuremath{\ \oplus}}
\newcommand{\exo}[1]{%
\addtocontents{toc}{\protect\setcounter{tocdepth}{2}}%
\paragraph{#1}}
\newcommand{\nullspace}[1]{\mathrm{null}\,{#1}}
\newcommand{\rangespace}[1]{\mathrm{range}\,{#1}}
\newcommand{\inner}[1]{\langle {#1}\rangle}
\newcommand{\norm}[1]{\|#1\|}
\newcommand{\jota}{\mathrm{j}}

% From DLL

% A4 reasonable margins (ie not wasting margins).

\usepackage{calc} % dimensions, tipically
\usepackage{subfigure}
\usepackage{fancyhdr} % header-footer
\usepackage{fancybox} % boxes

\setlength{\paperwidth}{160mm}
\setlength{\paperheight}{250mm}
\setlength{\hoffset}{-1in} % Compensate the driver margins
\setlength{\voffset}{-1in} % Compensate the driver margins
\setlength{\oddsidemargin}{5mm} % original 20 25 % then 15 45
\setlength{\evensidemargin}{5mm}
%\setlength{\evensidemargin}{.382\paperwidth-\oddsidemargin}
\newlength{\saveoddsidemargin}\setlength{\saveoddsidemargin}{\oddsidemargin}
\newlength{\saveevensidemargin}\setlength{\saveevensidemargin}{\evensidemargin}
% original 20 25 % then 15 40
\newlength{\headoddsidemargin}\setlength{\headoddsidemargin}{\oddsidemargin}
\newlength{\headevensidemargin}\setlength{\headevensidemargin}{\oddsidemargin}
\setlength{\textwidth}{\paperwidth-\oddsidemargin-\evensidemargin}
\setlength{\headwidth}{\paperwidth-\headoddsidemargin-\headevensidemargin}
\setlength{\headheight}{15pt} % 15 pt
\setlength{\headsep}{10pt} % 25 pt
\setlength{\topmargin}{5mm}
\setlength{\footskip}{0mm}
\newlength{\bottommargin} % 20mm
\setlength{\bottommargin}{5mm}
\setlength{\textheight}{\paperheight-\topmargin-\headheight-\headsep-\footskip-\bottommargin}
\renewcommand{\baselinestretch}{1}
\flushbottom % same text height for all the pages


%: ->  foot

% These commented out because caption2 deprecated, would need replacements in caption.
%\setlength{\captionmargin}{10pt}
%\setlength{\abovecaptionskip}{10pt}
%\setlength{\belowcaptionskip}{0pt}
%\renewcommand{\captionfont}{\sffamily}
%\makeatletter
%\newcommand\figcaption{%
%	\renewcommand{\figurename}{Fig.}\def\@captype{figure}\caption}
%\newcommand\tabcaption{\def\@captype{table}\caption}
%\makeatother

%: -> figures

\renewcommand{\subfigtopskip}{10pt}
\renewcommand{\subfigcapskip}{5pt}
\renewcommand{\subfigbottomskip}{10pt}
\newcommand{\goodgap}{%
	\hspace{\subfigtopskip}%
	\hspace{\subfigbottomskip}}

%: -> line spacing

\sloppy % imprescindible...
\frenchspacing
\setlength{\parindent}{2em}
\setlength{\parskip}{0ex}

%: -> to adjust properly the figure

\renewcommand{\floatpagefraction}{.8}
\newcommand{\topnumber}{2}
\renewcommand{\topfraction}{.8}
\renewcommand{\bottomfraction}{.8}
\renewcommand{\textfraction}{0.0}

%: -> fancy pagestyle

\pagestyle{fancy}
\renewcommand{\headrulewidth}{0pt}
\fancyhead[RO]{\flushright\thepage}
\fancyhead[LO]{\rightmark}
\renewcommand{\footrulewidth}{0.0pt}
\renewcommand{\sectionmark}[1]{%
	\markright{\thesection.\ #1}{}}
\fancyhf{}
\fancyhead[L]{\thepage}
\fancyhead[RO]{\thepage}
\fancyhead[LO]{\rightmark}
\fancyhead[R]{\sc\leftmark}
\fancypagestyle{plain}{\fancyhf{}}

\usepackage{parskip}
\linespread{0.95}
\interfootnotelinepenalty=10000

\selectlanguage{english}
\setcounter{secnumdepth}{10}
\setcounter{tocdepth}{1}


% http://tex.stackexchange.com/questions/1230/reference-name-of-description-list-item-in-latex
\makeatletter
\let\orgdescriptionlabel\descriptionlabel
\renewcommand*{\descriptionlabel}[1]{%
  \let\orglabel\label%
  \let\label\@gobble%
  \phantomsection%
  \edef\@currentlabel{#1}%
  \edef\@currentlabelname{#1}%
  \let\label\orglabel%
  \orgdescriptionlabel{#1}%
}
\makeatother


% MAIN DOCUMENT ----------------------------------------------------------------
\begin{document}
\title{From \emph{Linear Algebra Done Right}~\cite{Axler1997}}
\date{2015/9}
\date{2021/10}
\maketitle

\tableofcontents

\newpage
\setcounter{section}{0}


%%%%%%%%%%%%%%%%%%%%%%%%%%%%%%%%%%%%%%%%%%%%%%%%%%%%%%%%%%%%%%%
\input{axler.01.tex}


%%%%%%%%%%%%%%%%%%%%%%%%%%%%%%%%%%%%%%%%%%%%%%%%%%%%%%%%%%%%%%%
\newpage
\section{Finite-Dimensional Vector Spaces}
\subsection*{Summary}
\begin{itemize}
\item \textbf{linear combination} of a list $(v_1, \cdots, v_m)$ of vectors in $V$ is a vector $u = \sum a_i v_i$ with $a_i\in \mathbf{F}$
\item \textbf{span} of $(v_1, \cdots, v_m)$ is the set $\mathrm{span}(v_1, \cdots, v_m) = \{ \sum a_i v_i \mid a_i\in\mathbf{F}\}$
\begin{itemize}
  \item is the set of all linear combinations
  \item is a subspace of $V$
  \item for an empty list: $\mathrm{span}() = \{0\}$
  \item if $\mathrm{span}(v_1, \cdots, v_m) = V$ (i.e. $(v_1, \cdots, v_m)$ spans $V$ ) then $\mathrm{span}(v_1, \cdots, v_m)$ is \emph{the smallest subspace} in $V$
  \item a vector space $V$ is \textbf{finite dimensional} if exists a finite list $(v_1, \cdots, v_m)$ that spans $V$
\end{itemize}
\item A polynomial $p(z)$ 
\begin{itemize}
  \item has \textbf{degree} $m$ if $p(z)=a_0+a_1z+\cdots+a_mz^m$ with $a_i\in \mathbf{F}$ and $a_m\neq 0$
  \item $p(z)=0$ has degree $-\infty$
  \item $\mathcal{P}_m(\mathbf{F})$ set of all polynomials with degree $\leq m$
  \item $\mathcal{P}_m(\mathbf{F})$ is a finite dimensional subspace of $\mathcal{P}(\mathbf{F})$
\end{itemize}
\item none finite dimensional spaces are \textbf{infinite dimensional} ex. $\mathcal{P}(\mathbf{F})$ or $\mathbf{F}^\infty$
\item A list of vectors $(v_1,\cdots,v_n)$ is \textbf{linearly independent} in $V$ $\iff 0 = \sum a_i v_i$
and the only choice is $a_1=\cdots=a_n=0$. Otherwise is \textbf{linearly dependent}
\begin{itemize}
  \item a list $(v,)$ of length 1 is l.i. iff $v\neq 0$
  \item a list of length 2 is l.i. neither vector is scalar multiple of the other
  \item a list of length 3 or more is l.d. even thought no vector in the list is scalar multiple of another!
  \item removing vectors from a l.i. list remains l.i.
  \item we declare empty list $()$ as l.i.
\end{itemize}

\item[L2.4]\emph{Linear Dependence Lemma}: If $(v_1, \cdots, v_m)$ l.d. in $V$ and $v_1\neq 0$ then $\exists j\in 2,\cdots m$ such that
\begin{itemize}
  \item $v_j \in \mathrm{span}(v_1,\cdots, v_{j-1})$
  \item  $\mathrm{span}(v_1,\cdots, v_{j-1}, v_{j+1}, \cdots, v_m) = \mathrm{span}(v_1,\cdots, v_m)$
\end{itemize}
\item[T2.6:\label{it:T2_6}] In a \emph{finite} dimensional vector space, the \emph{length}(of every list of indepedent vectors) $\leq$ of the \emph{length}(of every spanning list of vectors)
\item[P2.7:] Every subspace $S$ of a \emph{finite}-dimensional vector space $V$ (i.e. $S\subset V$) is also \emph{finite}-dimensional
%

\item A \textbf{basis} of a vector space $V$ is a list of vectors in $V$ that is \emph{linearly indepedent} and \emph{spans} $V$.
\begin{itemize}
\item $(e_1, \cdots, e_n)$ where $e_i=(0,\cdots, \underbrace{1}_{i},\cdots,0)\in\mathbf{F}^n$ is the \textbf{standard basis} of $\mathbf{F}^n$. 
\item By \tref{it:T2_6}, a basis of $V$ is the smallest list of l.i. vectors that spans $V$
\end{itemize}
\item[P8:] A list $(v_1,\cdots,v_n)$ of vectors in $V$ is a basis of $V$ $\iff$ $\forall v\in V$ can be written uniquely as a linear combination of these vectors $v=\sum a_i v_i$.
\item[T10/12:] Every \emph{spanning/linearly independent}  list of vectors in a finite-dimensional vector space can be \emph{reduced}/\emph{extended} to a basis of the vector space.
\item[C11:] Every finite dimensional vector space \emph{has} a basis.
\item[P13:] $V$ finite dimensional and $U$ subspace of $V$ $\implies$ there is a subspace $W$ of $V$ such that $V=U\oplus W$.
%
\item[T14:] Any two bases of a finite-dimensional vector space have the same length
\item The \textbf{dimension} of a finite-dimensional vector space $\dimension{V}$ is the \emph{length} of any basis of the vector space
\begin{itemize}
  \item $\dimension{\mathbf{F}^n} = n$
  \item $\mathcal{P}(\mathrm{F}_m) = m+1$
\end{itemize}
\item[P15:] if $U$  subspace of finite-dimensional $V$ $\implies$ $\dimension{U}\leq\dimension{V}$.
%
\item[P16/17:]Every spanning/linearly-independent list of vectors in $V$ with the right length ($=\dimension{V}$) is a basis of $V$ 
%
\item[T18:] If $U_1,U_2$ subspaces of a finite-dimensional vector space then 
\begin{align*}
\dimension(U_1+U_2)=\dimension(U_1)+\dimension(U_2) - \dimension(U_1\cap U_2)
\end{align*}
\item[P19:\label{it:T2_19}] $V$ finite dimensional vector space and $U_1,\cdots,U_m$ subspaces of $V$ such that 
\begin{itemize}
  \item $V=\sum U_i$
  \item $\dimension{V}=\sum\dimension{U_i}$
\end{itemize}
 then $V=U_1 \oplus \cdots \oplus U_n = \bigoplus U_i$
\end{itemize}



%------------------------------------------------------------------------ 
\subsection*{Exercises}
%------------------------------------------------------------------------ 
\exo{}
By definition, the $\mathrm{span}(v_1,\cdots,v_n)$ is a set of all linear combination of the list. Since 
the new list has the same lenght and every element is itself a l.c. of two of them $u_i = v_i-v_{i+1}$, 
then both sets are identical \qed

%------------------------------------------------------------------------ 
\exo{} If $(v_1,\cdots,v_n)$ is l.i. in $V$ then $\sum a_i v_i = 0$ and $a_i = 0\; \forall i$, so any l.c. of the new list 
\begin{align*}
  c_1 (v_1 - v_2) + c_2 (v_2 - v_3) + \cdots + c_n v_n &= 0\\
  c_1 v_1 + (c_2 - c_1) v_2 + \cdots (c_n - c_{n-1})v_n &= 0
\end{align*}
so $c_1=0$, $c_2 = c_1$, ...  so it is l.i. as well \qed
%------------------------------------------------------------------------ 
\exo{} Let $\sum a_i (v_i + w) = \sum a_i v_i + (\sum a_i) w = 0$ and since some $a_j\neq 0$ then $\sum a_i\neq 0$ 
which leads to $w = \sum \frac{-a_i}{\sum a_i} v_i$ \qed 

%------------------------------------------------------------------------ 
\exo{} No, since it is not close in addition. E.g. the sum of two of its elements $ (1 + z^m) + (- z^m) = 1  $ 
results in a polynomial with smaller degree. In contrast note that $\mathcal{P}(\mathrm{F}_m)$ (with degree $\leq$ m ) 
\emph{is} a subspace. 

%------------------------------------------------------------------------ 
\exo{}
\hilight{TODO}

%------------------------------------------------------------------------ 
\exo{}
\hilight{TODO}

%------------------------------------------------------------------------ 
\exo{}
\hilight{TODO}

%------------------------------------------------------------------------ 
\exo{} $\forall u\in U$ is given by $u = (3\alpha , \alpha, 7\beta, \beta, \lambda)$ with $\alpha,\beta,\lambda \in \mathbf{R}$, which can be written 
as $ u = \alpha \underbrace{(3 , 1, 0, 0, 0)}_{u_1} + \beta\underbrace{(0 , 0, 7, 1, 0)}_{u_2} + \lambda \underbrace{(0 , 0, 0, 0, 1)}_{u_3} $ therefore 
 $(u_1,u_2,u_3)$ span $U$. In addition, they are l.i. in $\mathbf{R}^5$ and therefore a base of $U$.

%------------------------------------------------------------------------ 
\exo{} Yes. The list ($1$, $z$, $z^3+z^2$, $z^3$) is a base of $\mathcal{P}_3(\mathbf{F})$ and none of the polynomials (vectors) have degree 2. Note that $z^2 = \mathbf{p}_2 - \mathbf{p}_3$.

%------------------------------------------------------------------------ 
\exo{} $V$ with $\mathrm{dim}V = n$ then there is a base $(v_1, \cdot, v_n)$ which by definition $u = \sum a_i v_i$ and $a_i$ are unique. 
Take each of $v_i$ to span a one dimensional subspace $U_i$, we clearly have that $ V = \bigoplus U_i $


%------------------------------------------------------------------------ 
\exo{}
\hilight{TODO}



%%%%%%%%%%%%%%%%%%%%%%%%%%%%%%%%%%%%%%%%%%%%%%%%%%%%%%%%%%%%%%%
\newpage
\section{Linear Maps}
%------------------------------------------------------------------------ 
\subsection*{Summary}

%%%%%%%%%%%%%%%%%%%%%%%%%%%%%%%%%%%%%%%%%%%%%%%%%%%%%%%%%%%%%%%
\newpage
\selectlanguage{english}
\small
\bibliography{refs}
\bibliographystyle{plain}

\end{document}
