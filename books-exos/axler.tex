% -*- Mode: LaTeX; tab-width: 4 -*-
%
% Work on Sheldon Axler 'Linear Algebra Done Right'.
% Pedro Crespo Valero - 2015, 2021

\documentclass[11pt,notitlepage,oneside]{article}
\author{Pedro Crespo-Valero}

\usepackage[margin=1.5cm]{geometry} % to tell the pdf about our paper size
\usepackage[pdfpagelabels,pdfpagemode=None,backref=page,pagebackref=true,
colorlinks=true,bookmarks=true,bookmarksnumbered=true]{hyperref}
\usepackage{nameref}
\usepackage[latin1]{inputenc} % accents, REQUIRED for non english languages
\usepackage[english]{babel}
\usepackage{graphicx} % anything beyond pure text requires this
\usepackage[centertags,sumlimits,nointlimits,reqno]{amsmath}
\usepackage{amssymb}
\usepackage{amsbsy}
\usepackage{amsxtra}
\usepackage{amsthm}
\usepackage{mathtools}
\usepackage[nottoc,notbib]{tocbibind} % to put the bibliography in the toc
\usepackage[square,comma,numbers,sort&compress]{natbib} % many options in bibl.
\usepackage{tikz}
\usetikzlibrary{calc}
\usetikzlibrary{shapes,backgrounds}
%\usepackage{color}
%\usepackage{enumitem}
\usepackage{pifont}% http://ctan.org/pkg/pifont
\usepackage[printonlyused,withpage]{acronym}

%% NEW COMMANDS ------
\newcommand{\cmark}{\ding{51}}%
\newcommand{\xmark}{\ding{55}}%
\newcommand{\hilight}[1]{\colorbox{yellow}{#1}}
\newcommand{\pfrac}[2]{\frac{\partial #1}{\partial #2}}
\newcommand{\diff}{\mathrm{d}}
\newcommand{\firstsaid}[1]{\textbf{#1}}
\newcommand{\relphantom}[1]{\mathrel{\phantom{#1}}}
\newcommand{\wedgeseq}{\wedge\ldots\wedge}
\newcommand{\hodge}{\star}
\newcommand{\laplacian}{\Delta}
\DeclareMathOperator{\rot}{curl}
\DeclareMathOperator{\divg}{div}
\DeclareMathOperator{\dimension}{dim}
\DeclareMathOperator{\grad}{grad}
\DeclareMathOperator{\trace}{trace}
\DeclareMathOperator{\conv}{conv}

\renewcommand{\theparagraph}{\arabic{section}.\arabic{paragraph}}
\newcommand{\exref}[1]{Ex.\ref{#1}}
\newcommand{\figref}[1]{Fig.\ref{#1}}
\newcommand{\pref}[1]{Pr.\ref{#1}}
\newcommand{\tref}[1]{Th.\ref{#1}}

\newcommand{\tbc}{\ensuremath{\ \oplus}}
\newcommand{\exo}[1]{%
\addtocontents{toc}{\protect\setcounter{tocdepth}{2}}%
\paragraph{#1}}
\newcommand{\nullspace}[1]{\mathrm{null}\,{#1}}
\newcommand{\rangespace}[1]{\mathrm{range}\,{#1}}
\newcommand{\inner}[1]{\langle {#1}\rangle}
\newcommand{\norm}[1]{\|#1\|}
\newcommand{\jota}{\mathrm{j}}

% From DLL

% A4 reasonable margins (ie not wasting margins).
% Originaly copied and modified from https://github.com/lloda

\usepackage{calc} % dimensions, tipically
\usepackage{subfigure}
\usepackage{fancyhdr} % header-footer
\usepackage{fancybox} % boxes

\setlength{\paperwidth}{160mm}
\setlength{\paperheight}{250mm}
\setlength{\hoffset}{-1in} % Compensate the driver margins
\setlength{\voffset}{-1in} % Compensate the driver margins
\setlength{\oddsidemargin}{5mm} % original 20 25 % then 15 45
\setlength{\evensidemargin}{5mm}
%\setlength{\evensidemargin}{.382\paperwidth-\oddsidemargin}
\newlength{\saveoddsidemargin}\setlength{\saveoddsidemargin}{\oddsidemargin}
\newlength{\saveevensidemargin}\setlength{\saveevensidemargin}{\evensidemargin}
% original 20 25 % then 15 40
\newlength{\headoddsidemargin}\setlength{\headoddsidemargin}{\oddsidemargin}
\newlength{\headevensidemargin}\setlength{\headevensidemargin}{\oddsidemargin}
\setlength{\textwidth}{\paperwidth-\oddsidemargin-\evensidemargin}
\setlength{\headwidth}{\paperwidth-\headoddsidemargin-\headevensidemargin}
\setlength{\headheight}{15pt} % 15 pt
\setlength{\headsep}{10pt} % 25 pt
\setlength{\topmargin}{5mm}
\setlength{\footskip}{0mm}
\newlength{\bottommargin} % 20mm
\setlength{\bottommargin}{5mm}
\setlength{\textheight}{\paperheight-\topmargin-\headheight-\headsep-\footskip-\bottommargin}
\renewcommand{\baselinestretch}{1}
\flushbottom % same text height for all the pages


%: ->  foot

% These commented out because caption2 deprecated, would need replacements in caption.
%\setlength{\captionmargin}{10pt}
%\setlength{\abovecaptionskip}{10pt}
%\setlength{\belowcaptionskip}{0pt}
%\renewcommand{\captionfont}{\sffamily}
%\makeatletter
%\newcommand\figcaption{%
%	\renewcommand{\figurename}{Fig.}\def\@captype{figure}\caption}
%\newcommand\tabcaption{\def\@captype{table}\caption}
%\makeatother

%: -> figures

\renewcommand{\subfigtopskip}{10pt}
\renewcommand{\subfigcapskip}{5pt}
\renewcommand{\subfigbottomskip}{10pt}
\newcommand{\goodgap}{%
	\hspace{\subfigtopskip}%
	\hspace{\subfigbottomskip}}

%: -> line spacing

\sloppy % imprescindible...
\frenchspacing
\setlength{\parindent}{2em}
\setlength{\parskip}{0ex}

%: -> to adjust properly the figure

\renewcommand{\floatpagefraction}{.8}
\newcommand{\topnumber}{2}
\renewcommand{\topfraction}{.8}
\renewcommand{\bottomfraction}{.8}
\renewcommand{\textfraction}{0.0}

%: -> fancy pagestyle

\pagestyle{fancy}
\renewcommand{\headrulewidth}{0pt}
\fancyhead[RO]{\flushright\thepage}
\fancyhead[LO]{\rightmark}
\renewcommand{\footrulewidth}{0.0pt}
\renewcommand{\sectionmark}[1]{%
	\markright{\thesection.\ #1}{}}
\fancyhf{}
\fancyhead[L]{\thepage}
\fancyhead[RO]{\thepage}
\fancyhead[LO]{\rightmark}
\fancyhead[R]{\sc\leftmark}
\fancypagestyle{plain}{\fancyhf{}}

\usepackage{parskip}
\linespread{0.95}
\interfootnotelinepenalty=10000

\selectlanguage{english}
\setcounter{secnumdepth}{10}
\setcounter{tocdepth}{1}


% http://tex.stackexchange.com/questions/1230/reference-name-of-description-list-item-in-latex
\makeatletter
\let\orgdescriptionlabel\descriptionlabel
\renewcommand*{\descriptionlabel}[1]{%
  \let\orglabel\label%
  \let\label\@gobble%
  \phantomsection%
  \edef\@currentlabel{#1}%
  \edef\@currentlabelname{#1}%
  \let\label\orglabel%
  \orgdescriptionlabel{#1}%
}
\makeatother


% MAIN DOCUMENT ----------------------------------------------------------------
\begin{document}
\title{From \emph{Linear Algebra Done Right}~\cite{Axler1997}}
\date{2015/9}
\date{2021/10}
\maketitle

\tableofcontents

\newpage
\setcounter{section}{0}
%%%%%%%%%%%%%%%%%%%%%%%%%%%%%%%%%%%%%%%%%%%%%%%%%%%%%%%%%%%%%%%

% -*- Mode: LaTeX; tab-width: 4 -*-
%
% Work on Sheldon Axler 'Linear Algebra Done Right'.
% Pedro Crespo Valero - 2015, 2021
%

%  Chapter 1

%%%%%%%%%%%%%%%%%%%%%%%%%%%%%%%%%%%%%%%%%%%%%%%%%%%%%%%%%%%%%%%
\section{Vector Spaces}
\subsection*{Summary}
% Summary -----------------------------------------------------
\begin{itemize}
\item \textbf{scalars} $\mathbb{F}$ (i.e. fields $\mathbb{R}$ or $\mathbb{C}$)
\begin{itemize}
\item commutativity +, $\cdot$
\item associativity +, $\cdot$
\item identity +0, $\cdot 1$
\item \emph{additive} inverse: $\exists!w \mid z + w = 0$, denoted $-z$
\item \emph{multiplicative} inverse: $\exists!w$ and $z\neq0$ then $z\cdot w = 1$, denoted $1/w$
\item distributive: $a(b+c) = ab+ac$
\end{itemize}
%---
\item A \textbf{vector space} is a set $V$ along with an \emph{addition}  and a \emph{scalar multiplication} on $V$ that satisfies:
\begin{itemize}
\item commutativity +
\item associativity +
\item \emph{additive} identity: $0\in V$ such that $v+0=v$ $\forall v\in V$
\item \emph{additive} inverse: $\forall v\in V,\; \exists!w\in V$ such that $v+w=0$, denoted $-v$
\item \emph{multiplicative} identify: $1v=v\;\forall v\in V$ and $1\in\mathbb{F}$
\item distributive: $a(u+v)=au+av$ and $(a+b)u=au+bu$ $\forall a,b\in\mathbb{F}$ and $\forall u,v \in V$.
\end{itemize}
 
\item[P1:\label{it:P1_1}] A \emph{vector space} has a \emph{unique} additive identity $0$
\item[P2:\label{it:P1_2}] Every \emph{element} of a vector space has a \emph{unique} additive inverse $v\leftarrow -v$
\item[P3:] $0v=0$ $\forall v\in V$
\item[P4:] $a0=0$ $\forall a\in \mathbb{F}$
\item[P5:] $(-1)v = -v$ $\forall v\in V$, i.e. the additive inverse

\item \textbf{subspace}: $U\subset V$ is a subspace of $V$ if $U$ is also a vector space, that is
\begin{itemize}
\item has additive identity: $0\in U$
\item closed under addition $\forall u,v \in V$: $u+v \in U$ 
\item closed under scalar multiplication $\forall \lambda\in \mathbb{F}$ and $\forall u \in V$: $\lambda u \in U$  
\item remaining conditions for a vector space are automatically satisfied since $U\subset V$.
\item Examples: \emph{the only} subspaces of ...
\begin{itemize}
  \item $\mathbf{R}^2$ are $\{0\}, \mathbf{R}^2$ and all lines in $\mathbf{R}^2$ through the origin
  \item $\mathbf{R}^3$ are $\{0\}, \mathbf{R}^3$ and all lines, and all planes in $\mathbf{R}^2$ through the origin
\end{itemize}

\end{itemize} 

\item \textbf{sum of subspaces}: $U_1+\cdots U_n = \sum U_i =\{\sum u_i\mid u_i\in U_i\}$ and $U_i\subset V$ subspaces of $V$. $\sum U_i$ is
\begin{itemize}
\item a subspace of $V$.
\item the smallest subspace of $V$ that contains all $U_i$.
\end{itemize}

\item $V$ is a \textbf{direct sum of subspaces} $U_i$, written $V=U_1\oplus\cdots\oplus U_n$, if each element of $V$ can be written \emph{uniquely} as a sum $\sum u_i$, where each $u_i\in U_i$. 
Then $V=\bigoplus U_i \iff$
\begin{itemize}
\item $V = \sum U_i$
\item $0=\sum u_i$ and $u_i=0 \in U_i$.
\end{itemize}

\item[P6:\label{it:P1_6}] $U,W$ subspaces of $V$ and $V=U\oplus W \iff$
\begin{itemize}
\item $V=U+V$
\item $U\cap V=\{0\}$
\end{itemize}
but \emph{only} valid for \emph{pairs} of subspaces!

\end{itemize}

\subsection*{Exercises}
%------------------------------------------------------------------------ 
\exo{}
\begin{align*}
\frac{1}{z} = 1 \frac{1}{z} = (z^* \cdot \frac{1}{z^*}) \frac{1}{z} = \frac{z^*}{zz^*}=\frac{a-\mathrm{j}b}{a^2 + b^2} = c+\mathrm{j}d
\end{align*}
where we use first identity, then multiplicative inverse in $\mathbb{C}$, and finally by replacing $z=a+jb$, we use distributive property 
to get $zz^* = a^2 + b^2 $\qed

%------------------------------------------------------------------------ 
\exo{}
\begin{align*}
\sqrt[3]{1} = \exp{\left(\mathrm{j}\frac{2\pi k}{3}\right)} = \cos(2\pi k/3) +\mathrm{j}\sin(2\pi k/3) = \left\{ 1, \frac{-1+\sqrt{3}\mathrm{j}}{2}, \frac{-1-\sqrt{3}\mathrm{j}}{2} \right\} 
\end{align*}
for $k=0,1,2$.  \qed
%------------------------------------------------------------------------ 
\exo{} Applying P5 twice and the multiplicative identity
\begin{align*}
-(-v)= (-1)(-1)v = 1v = v  
\end{align*}\qed
%------------------------------------------------------------------------ 
\exo{} \textit{Reductio ad absurdum}: Let $av=0$ with $a\neq 0 \in \mathbb{F}$, $v\neq0 \in V$. Using the scalar's multiplicative inverse property, if $a\neq 0$ there exists a unique $b\in \mathbb{F}$ that inverses it $ab = ba = 1$. Then
\begin{align*}
av &= 0 & \\ 
bav &= b0 & \\
1v &= 0 & \text{mult. inverse \& conmutative + P3}\\ 
v &=0 & \text{mult. identity}
\end{align*}
which is a contradiction. 
\qed
%------------------------------------------------------------------------ 
\exo{}
\begin{enumerate}
\item[a)] Yes. $V=\{x\in\mathbb{F}^3\mid vx^T=0$ with $v=(1,2,3)\in \mathbb{F}^3 \}$. A plane through the origin is a subspace over $\mathbb{F}^3$ since $\forall x_1, x_2 \in V$
\begin{itemize}
\item Contains $0$, since $v0^T= 0$
\item Closed under $+$: $v(x_1^T + x_2^T) = vx_1^T + vx_2^T= 0+0 = 0$ (distributive )
\item Closed under scalar $\times$: $v \lambda x^T = \lambda (v x^T) = \lambda 0 = 0$  (commutative and P4)
\end{itemize}

\item[b)] No. $V=\{x\in\mathbb{F}^3\mid vx^T=b\}$ with  $v=(1,2,3)$ and $b=4$ is an affine set that does not contain $0$ since $v0\neq 4$. Therefore is \emph{not} a subspace. Can also be written as $v(x-x_0)^T = 0$ or $(x_1-4) + 2x_2 + 3x_3=0$.

\item[c)] No.  $V=\{x\in\mathbb{F}^3\mid x_1x_2x_3=0\}$ is the \emph{union} of three planes $x_i=0$ (since either $x_1=0$ or $x_2=0$ or $x_3=0$ are solutions). 
This case includes $0$ but is not closed under sum because $\forall x,y\in V$ then $\Pi_i(x_i+y_i)-\Pi_i(x_i)-\Pi_i(y_i) \neq 0$. 
Each plane independently $x_i=0$  is a subspace but its union (careful, not it's \emph{sum}) in $V$ is \emph{not} a subspace.

\item[d)]Yes. $V=\{x\in\mathbb{F}^3\mid x_1=5x_3\}$ is a line through the origin and is a subspace of $\mathbb{F}^3$:
\begin{itemize}
\item $0\in V$
\item closed under $+$: $ x_1+y_1 = 5x_3+5y_3 = 5(x_3+y_3)$
\item closed under scalar $\times$: $\lambda x_1 = \lambda 5x_3$
\end{itemize}

\end{enumerate}
%------------------------------------------------------------------------ 
\exo{}
$V = \mathbb{I}^2 \in \mathbb{R}^2$ where every point is a pair of integers $x=(m,n)$. This case includes additive identity, is closed under sum since the sum of integers are integers, also under additive inverses since subtracting integers gives integers but \emph{not} closed under scalar multiplication since an integer times a real number is in general real.

%------------------------------------------------------------------------ 
\exo{} Intuitively, we need to find a set with "a shape" that can scale and does not change (i.e. close in scalar multiplication). 
For instance, take "a cross" $ U = \{ (x,y) \mid x.y = 0 \}$ is close in scalar multiplication, includes $\{0\}$ but is \emph{not} close in addition
therefore is not a subspace or $\mathbb{F}^2$.


%------------------------------------------------------------------------ 
\exo{}$\cap U_i$ and $U_i$ subspaces of $V$. 
\begin{itemize}
\item $0 \in \cap U_i$ since $0\in U_i$
\item $x,y \in \cap U_i \implies x,y \in U_i\; \forall i \implies x+y \in U_i\; \forall i \implies x+y\in \cap U_i$
\item analogous scalar product
\end{itemize}
therefore $\cap U_i$ is a subspace of $V$.\qed

%------------------------------------------------------------------------ 
\exo{} Let $A,B$ subspaces of $V$
\begin{itemize}
\item[$\leftarrow$)]$A\subset B \subset V$ then $A\cup B = B$ which is already a subspace of $V$.
\item[$\rightarrow$)] $A\cup B$ is a subspace $V$.  $a+b \in B$ with $a\in A$, $b\in B$. Then $ (a + b) - b \in B$ because $-b\in B$, but then $a + b - b = a \in B$ so $A\in B$.
\end{itemize}

%------------------------------------------------------------------------ 
\exo{}$U+U =\{ u+v \mid \forall u,v \in U\}$ since $U$ is a subspace of $V$ then $u+v\in U$ therefore $U+U=U$.

%------------------------------------------------------------------------ 
\exo{} yes since the sum of subspaces is a subspace and all elements in the subspace satisfy commutativity and associativity. 

%------------------------------------------------------------------------ 
\exo{} \begin{itemize}
\item $\sum U_i$ is a subspace and any subspace, by \pref{it:P1_1}, has a unique additive identity. Therefore $\sum U_i + \sum 0 = \sum U_i$.
\item  Is there $V$ such that $U+V = \{0\}$? Only the subspace $\{0\}$ because if we make a set of all the additive inverses of a subspace $-U = {-u \mid u\in U}$ then we end up with the subspace itself: $ U - U = U$. 
\end{itemize}
%------------------------------------------------------------------------ 
\exo{} No. For instance, since any line through the origin is a subspace of $\mathbf{R}^2$, then by \pref{it:P1_6}, we also know 
that $\mathbf{R}^2$ is a direct sum of any pair of these lines (subspaces) $U_i, W$. 
An example is
\begin{align*}
  U_1 &= \{(x,0)\in \mathbb{R}^2 :x \in \mathbb{R} \}  \\
  U_2 &= \{(x,x)\in \mathbb{R}^2 :x \in \mathbb{R} \}  \\
  W   &= \{(0,y)\in \mathbb{R}^2 :y \in \mathbb{R} \}  \\
\end{align*}
and $\mathbf{R}^2 = U_1 \oplus W = U_2 \oplus W $ .

%------------------------------------------------------------------------ 
\exo{} Let $W$ be a subspace of $\mathcal{P}(\mathbf{F})$ consisting of all polynomials as
\begin{align}
  q(z) = \sum\limits_{i\neq2,5} c_i z
\end{align}
then $(p+q)(z) = c_0 + c_1z + a z^2+ c_3 z^3 + c_4 z^4 + b z^5 + c_6 z^6 + ...$ and therefore $\mathcal{P}(\mathbf{F}) = U \oplus W$. 

%------------------------------------------------------------------------ 
\exo{} No. Counterexample: take three lines as subspaces of $V=\mathbf{R}^2$ such as $U_1=\{(x,0)\}, U_2=\{(x,x)\}, W=\{(0,x)\}$ with $x\in \mathbf{R}$. Since
$U_i\cap W = \{0\}$ then by \pref{it:P1_6}, $\mathbf{R}^2$ is direct sum of subspaces $U_1,W$ or $U_2,W$ but still $U_1\neq U_2$. 
%------------------------------------------------------------------------ 

%%%%%%%%%%%%%%%%%%%%%%%%%%%%%%%%%%%%%%%%%%%%%%%%%%%%%%%%%%%%%%%

% -*- Mode: LaTeX; tab-width: 4 -*-
%
% Work on Sheldon Axler 'Linear Algebra Done Right'.
% Pedro Crespo Valero - 2015, 2021

%  Chapter 2

%%%%%%%%%%%%%%%%%%%%%%%%%%%%%%%%%%%%%%%%%%%%%%%%%%%%%%%%%%%%%%%

\section{Finite-Dimensional Vector Spaces}
\subsection*{Summary}
\begin{itemize}
\item \textbf{linear combination} of a list $(v_1, \cdots, v_m)$ of vectors in $V$ is a vector $u = \sum a_i v_i$ with $a_i\in \mathbf{F}$
\item \textbf{span} of $(v_1, \cdots, v_m)$ is the set $\mathrm{span}(v_1, \cdots, v_m) = \{ \sum a_i v_i \mid a_i\in\mathbf{F}\}$
\begin{itemize}
  \item is the set of all linear combinations
  \item is a subspace of $V$
  \item for an empty list: $\mathrm{span}() = \{0\}$
  \item if $\mathrm{span}(v_1, \cdots, v_m) = V$ (i.e. $(v_1, \cdots, v_m)$ spans $V$ ) then $\mathrm{span}(v_1, \cdots, v_m)$ is \emph{the smallest subspace} in $V$
  \item a vector space $V$ is \textbf{finite dimensional} if exists a finite list $(v_1, \cdots, v_m)$ that spans $V$
\end{itemize}
\item A polynomial $p(z)$ 
\begin{itemize}
  \item has \textbf{degree} $m$ if $p(z)=a_0+a_1z+\cdots+a_mz^m$ with $a_i\in \mathbf{F}$ and $a_m\neq 0$
  \item $p(z)=0$ has degree $-\infty$ ???
  \item $\mathcal{P}_m(\mathbf{F})$ set of all polynomials with degree $\leq m$
  \item $\mathcal{P}_m(\mathbf{F})$ is a finite dimensional subspace of $\mathcal{P}(\mathbf{F})$
\end{itemize}
\item none finite dimensional spaces are \textbf{infinite dimensional} ex. $\mathcal{P}(\mathbf{F})$ or $\mathbf{F}^\infty$
\item A list of vectors $(v_1,\cdots,v_n)$ is \textbf{linearly independent} in $V$ $\iff 0 = \sum a_i v_i$
and the only choice is $a_1=\cdots=a_n=0$. Otherwise is \textbf{linearly dependent}
\begin{itemize}
  \item a list $(v,)$ of length 1 is l.i. iff $v\neq 0$
  \item a list of length 2 is l.i. neither vector is scalar multiple of the other
  \item a list of length 3 or more is l.d. even thought no vector in the list is scalar multiple of another!
  \item removing vectors from a l.i. list remains l.i.
  \item we declare empty list $()$ as l.i.
\end{itemize}

\item[L4]\emph{Linear Dependence Lemma}: If $(v_1, \cdots, v_m)$ l.d. in $V$ and $v_1\neq 0$ then $\exists j\in 2,\cdots m$ such that
\begin{itemize}
  \item $v_j \in \mathrm{span}(v_1,\cdots, v_{j-1})$
  \item  $\mathrm{span}(v_1,\cdots, v_{j-1}, v_{j+1}, \cdots, v_m) = \mathrm{span}(v_1,\cdots, v_m)$
\end{itemize}
\item[T6:\label{it:T2_6}] In a \emph{finite} dimensional vector space, the \emph{length}(of every list of indepedent vectors) $\leq$ of the \emph{length}(of every spanning list of vectors)
\item[P7:] Every subspace $S$ of a \emph{finite}-dimensional vector space $V$ (i.e. $S\subset V$) is also \emph{finite}-dimensional
%

\item A \textbf{basis} of a vector space $V$ is a list of vectors in $V$ that is \emph{linearly indepedent} and \emph{spans} $V$.
\begin{itemize}
\item $(e_1, \cdots, e_n)$ where $e_i=(0,\cdots, \underbrace{1}_{i},\cdots,0)\in\mathbf{F}^n$ is the \textbf{standard basis} of $\mathbf{F}^n$. 
\item By \tref{it:T2_6}, a basis of $V$ is the smallest list of l.i. vectors that spans $V$
\end{itemize}
\item[P8:] A list $(v_1,\cdots,v_n)$ of vectors in $V$ is a basis of $V$ $\iff$ $\forall v\in V$ can be written uniquely as a linear combination of these vectors $v=\sum a_i v_i$.
\item[T10/12:] Every \emph{spanning/linearly independent}  list of vectors in a finite-dimensional vector space can be \emph{reduced}/\emph{extended} to a basis of the vector space.
\item[C11:] Every finite dimensional vector space \emph{has} a basis.
\item[P13:] $V$ finite dimensional and $U$ subspace of $V$ $\implies$ there is a subspace $W$ of $V$ such that $V=U\oplus W$.
%
\item[T14:] Any two bases of a finite-dimensional vector space have the same length
\item The \textbf{dimension} of a finite-dimensional vector space $\dimension{V}$ is the \emph{length} of any basis of the vector space
\begin{itemize}
  \item $\dimension{\mathbf{F}^n} = n$
  \item $\mathcal{P}(\mathrm{F}_m) = m+1$
\end{itemize}
\item[P15:] if $U$  subspace of finite-dimensional $V$ $\implies$ $\dimension{U}\leq\dimension{V}$.
%
\item[P16/17] If a list has the right lenght, only need to verify one of the props to be base: either span or l.i.:
\begin{itemize}
  \item[P16:\label{it:T2_16}]Every spanning list of vectors in $V$ reduced to a length $\dimension{V}$ is a basis of $V$ 
  \item[P17:\label{it:T2_17}]Every linearly-independent list of vectors in $V$ extended to a lenght $\dimension{V}$ is a basis of $V$
\end{itemize}
%
\item[T18:\label{it:T2_18}] If $U_1,U_2$ subspaces of a finite-dimensional vector space then 
\begin{align*}
\dimension(U_1+U_2)=\dimension(U_1)+\dimension(U_2) - \dimension(U_1\cap U_2)
\end{align*}
\item[P19:\label{it:T2_19}] $V$ finite dimensional vector space and $U_1,\cdots,U_m$ subspaces of $V$ such that 
\begin{itemize}
  \item $V=\sum U_i$
  \item $\dimension{V}=\sum\dimension{U_i}$
\end{itemize}
 then $V=U_1 \oplus \cdots \oplus U_n = \bigoplus U_i$
\end{itemize}



%------------------------------------------------------------------------ 
\subsection*{Exercises}
%------------------------------------------------------------------------ 
\exo{}
\begin{align*}
a_1(v_1 - v_2) + a_2 (v_2 - v_3) + \cdots + a_{n-1} (v_{n-1} - v_n) + v_n = \\
a_1 v_1 + (a_1 - a_2) v_2 + \cdots + (1 - a_{n-1}) v_n 
\end{align*}
since the latter spans V, then $0 \neq a_1\neq a_2 \cdots$ 
\qed

%------------------------------------------------------------------------ 
\exo{} If $(v_1,\cdots,v_n)$ is l.i. in $V$ then $\sum a_i v_i = 0$ and $a_i = 0\; \forall i$, so any l.c. of the new list 
\begin{align*}
  c_1 (v_1 - v_2) + c_2 (v_2 - v_3) + \cdots + c_n v_n &= 0\\
  c_1 v_1 + (c_2 - c_1) v_2 + \cdots (c_n - c_{n-1})v_n &= 0
\end{align*}
so $c_1=0$, $c_2 = c_1$, ...  so it is l.i. as well \qed
%------------------------------------------------------------------------ 
\exo{} Let $\sum a_i (v_i + w) = \sum a_i v_i + (\sum a_i) w = 0$ and since some $a_j\neq 0$ then $\sum a_i\neq 0$ 
which leads to $w = \sum \frac{-a_i}{\sum a_i} v_i$ \qed 

%------------------------------------------------------------------------ 
\exo{} No, since it is not close in addition. E.g. the sum of two of its elements $ (1 + z^m) + (- z^m) = 1  $ 
results in a polynomial with smaller degree. In contrast note that $\mathcal{P}(\mathrm{F}_m)$ (with degree $\leq m$ ) 
\emph{is} a subspace. 

%------------------------------------------------------------------------ 
\exo{}
\hilight{TODO}

%------------------------------------------------------------------------ 
\exo{}
\hilight{TODO}

%------------------------------------------------------------------------ 
\exo{}
\hilight{TODO}

%------------------------------------------------------------------------ 
\exo{} $\forall u\in U$ is given by $u = (3\alpha , \alpha, 7\beta, \beta, \lambda)$ with $\alpha,\beta,\lambda \in \mathbf{R}$, which can be written 
as $ u = \alpha \underbrace{(3 , 1, 0, 0, 0)}_{u_1} + \beta\underbrace{(0 , 0, 7, 1, 0)}_{u_2} + \lambda \underbrace{(0 , 0, 0, 0, 1)}_{u_3} $ therefore 
 $(u_1,u_2,u_3)$ span $U$. In addition, they are l.i. in $\mathbf{R}^5$ and therefore a base of $U$.

%------------------------------------------------------------------------ 
\exo{} Yes. The list ($1$, $z$, $z^3+z^2$, $z^3$) is a base of $\mathcal{P}_3(\mathbf{F})$ and none of the polynomials (vectors) have degree 2. Note that $z^2 = \mathbf{p}_2 - \mathbf{p}_3$.

%------------------------------------------------------------------------ 
\exo{} $V$ with $\mathrm{dim}V = n$ then there is a base $(v_1, \cdots, v_n)$ which by definition $u = \sum a_i v_i$ and $a_i$ are unique. 
Take each of $v_i$ to span a one dimensional subspace $U_i$, we clearly have that $ V = \bigoplus U_i $


%------------------------------------------------------------------------ 
\exo{}  Let $B_1$ bases of subspace $U\subseteq V $, then the list $B_1$ is also l.i. in $V$ with length $\dimension{U} = \dimension{V}$ 
therefore, by \pref{it:T2_17}, is also a bases of $V$ which means that both are the same set $U=V$. 


%------------------------------------------------------------------------ 
\exo{} $a_i=0$ is not the only solution $\forall z\in \mathbf{F}$ for $\sum a_i \mathbf{p}_i(z) = 0$ since at $z=2$
all polynomials have a root, i.e. $\mathbf{p}_i(2)=0$.  Therefore it cannot be linearly indepedent. 
\hilight{???}

%------------------------------------------------------------------------ 
\exo{} By \tref{it:T2_18}, 
\begin{align*}
  \underbrace{\dimension{R^8}}_8 = \dimension{U+W} = \underbrace{\dimension{U}}_3 + \underbrace{\dimension{W}}_5 - \dimension{U \cap W} \implies \dimension{U \cap W} = 0
\end{align*}
which leads to the only subspace dimensionless, i.e. $U \cap W = \{0\}$
%------------------------------------------------------------------------ 
\exo{} Following previous reasoning leads to $ \dimension{U \cap W} = 1 \implies U\cap W \neq \{0\}$

%------------------------------------------------------------------------ 
\exo{} No. Take $u_1,u_2,u_3$ three vectors in $\mathbf{R}^2$ and bases spanning $U_1, U_2, U_3$ resp., then 
\begin{enumerate}
  \item $\dimension{(U_1+U_2+U_3)} = \dimension{\mathbf{R}^2} = 2$
  \item $\dimension{U_i} = 1$
  \item $U_i\cap U_j = \{0\}$ for $i\neq j$ 
\end{enumerate} 
but then $\dimension{(U_1+U_2+U_3)} \neq \dimension{U_1} + \dimension{U_2} + \dimension{U_3} - 0 - ... + 0$.



%------------------------------------------------------------------------ 
\exo{}  Using \tref{it:T2_18} we have that $\dimension{  \sum U_i  } = \sum \dimension{U_i} - \sum \underbrace{\dimension{ (U_i \cap \sum_{j>i} U_j) }}_{\ge 0}  $
$\implies  \dimension{  \sum U_i  } \leq \sum \dimension{U_i}  $

%------------------------------------------------------------------------ 

\exo{} Following from previous result, $U_i \cap \bigoplus\limits_{j>i} U_j = \{0\}$ because the bases of each $U_i$ are l.i. .



%%%%%%%%%%%%%%%%%%%%%%%%%%%%%%%%%%%%%%%%%%%%%%%%%%%%%%%%%%%%%%%
\newpage
\section{Linear Maps}
%------------------------------------------------------------------------ 
\subsection*{Summary}

\begin{description}
  \item[Linear Map] $T:V\to W$ with the following properties
  \begin{itemize}
  \item additivity $T(u+v) = Tu +Tv \in W$ for all $u,v\in V$
  \item homogeneity $T(av) = aTv \in W$ for all $a\in \mathbf{F}$
  \end{itemize}
  %
  \item[Set of Linear Maps\label{itm:D3_set_linear_maps}] $T\in \mathcal{L}(V,W)$
  \begin{itemize}
  \item $0$ map: $0v = 0$ and $0\in \mathcal{L}(V,W)$
  \item $I$ map: $Iv=v$ and $I\in \mathcal{L}(V,V)$
  \item differentiation map : $Tp = p'$ and $T\in \mathcal{L}(\mathcal{P}(\mathbf{R}),\mathcal{P}(\mathbf{R}))$
  \item integration map: $Tp = \int\limits_0^1p(x)dx$ and $T\in \mathcal{L}(\mathcal{P}(\mathbf{R}),\mathbf{R})$
  \item multiplication by $x^2$ map: $Tp=x^2p(x)$ with $T\in \mathcal{L}(\mathcal{P}(\mathbf{R}),\mathcal{P}(\mathbf{R}))$
  \item backwards shift map: $T x = T(x_1,x_2, \cdots) = (x_2, x_3, \cdots)$ with $T\in \mathcal{L}(\mathbf{F}^\infty, \mathbf{F}^\infty)$
  \item map $T:\mathbf{F}^n \to \mathbf{F}^m$: 
  \begin{align*}
  T x = T(x_1,\cdots,x_n) =  (T_1x,  \cdots, T_m x) = (\sum a_{1i}x_i,  \cdots, \sum a_{mi}x_i) = y
  \end{align*} with $a_{ij}\in\mathbf{F}$.
  
  \item $(v_1,\cdots,v_n)$ a basis of $V$ and $w_1,\cdots,w_m$ any choice of $W$. We construct a linear map such that $Tv_i=w_j$. Hence $\forall v\in V$
  \begin{align*}
  Tv = T(\sum a_i v_i) = \sum a_i w_i
  \end{align*}
  \end{itemize}
  
  \item[Vector Space] $\mathcal{L}(V,W)$  by \emph{defining} the following operations
  \begin{itemize}
  \item $(S+T)v = Sv + Tv \in W$ 
  \item $(aT)v = a(Tv) \in W$
  \end{itemize}
  $\forall S,V \in \mathcal{L}(V,W)$ and $a\in\mathbf{F}$ it satisfies
  \begin{itemize}
  \item closed under sum: $S+V \in \mathcal{L}(V,W)$
  \item closed under scalar multiplication: $aT \in \mathcal{L}(V,W)$
  \item additive identity is the zero map: $T+0=T$
  \end{itemize}
  %
  \item[Multiplication] (=composition) of linear maps: Let $U \xrightarrow{T} V \xrightarrow{S} W$ then we define
  \begin{align*}
  (ST)v = S(Tv)
  \end{align*}
  has the properties of
  \begin{itemize}
  \item associativity: $(T_1T_2)T_3 = T_1(T_2T_3)$.
  \item identity: $TI = T = IT$.
  \item distributive: $(S_1+S_2)T = S_1T+S_2T$ and $S(T_1+T_2) = ST_1 +ST_2$ for $U\xrightarrow{S,S_1,S_1} V \xrightarrow{T,T_1,T_2}$.
  \item \emph{not} conmutative: in general $ST \neq TS $
  \end{itemize}
  %
  \item[null space\label{itm:D3_nullspace}] of $T$: $\nullspace{T} = \{v\in V \mid Tv=0\}$
  \item[range] (or image) of $T$: $\rangespace{T} = \{Tv \mid v\in V\}$
  \item[P1]If $T\in\mathcal{L}(V,W)$ then $\nullspace{T}$ is a subspace of $V$
  \item[P3]If $T\in\mathcal{L}(V,W)$ then $\rangespace{T}$ is a subspace of $W$
  \item[P2\label{itm:T3_2}]$T\in\mathcal{L}(V,W)$ \textbf{injective} $\iff \nullspace{T}=\{0\}$
  \begin{itemize}
  \item $T:V\to W$ is injective if $\forall u,v \in V$ and $Tu=Tv$ then $u=v$
  \end{itemize}
  \item[surjective] $T\in\mathcal{L}(V,W)$ is surjective $\iff  \rangespace{T} = W$ (by definition)
  \item[T4] If $V$ is finite dimensional and $T\in\mathcal{L}(V,W)$, then
  \begin{itemize}
  \item $\rangespace{T}$ is finite dimensional
  \item the dimension of $V$ can be expressed as
  \begin{align*}
  \Aboxed{\dim{V} = \dimension{\nullspace{T}} + \dimension{\rangespace{T}}}
  \end{align*}
  \end{itemize}
  %
  \item$V,W$ finite dimensional vector spaces
  \begin{itemize}
  \item[C5] $\dimension{V} > \dimension{W} \implies \nexists\,T\in\mathcal{L}(V,W)$ injective
  \item[C6] $\dimension{V} < \dimension{W} \implies \nexists\,T\in\mathcal{L}(V,W)$ surjective
  \end{itemize}
  %
  \item Theory of linear equations. Define $T:\mathbf{F}^n\to\mathbf{F}^m$ by 
  \begin{align*}
  Tx = T(x_1,\cdots,x_n) = y = (\sum_k a_{1k}x_k, \cdots \sum_k a_{mk}x_k)
  \end{align*}
  then the system of \emph{homogeneous} equations can be written as $Tx=0$ and
  \begin{itemize}
  \item Solutions are $x\in\nullspace{T}$.
  \item $\dimension{\nullspace{T}}>0$ iff $T$ not injective.
  \item  $T$ not injective if $\dimension{V} > \dimension{W}$ or $n>m$ (i.e. number of unknowns > equations).
  \end{itemize}
  The system of \emph{inhomogeneous} equations can be written as $Tx=c$
  \begin{itemize}
  \item $\forall c\in\rangespace{T}$ exists a solution in $x$
  \item but every solution $c\in \mathbf{F}^m$, i.e. is $\rangespace{T}=\mathbf{F}^m$?
  \item only if $T$ surjective.
  \item $T$ not surjective if $\dimension{V} < \dimension{W}$ or $n<m$ (i.e. smaller number of unknowns than equations).
  \item An inhomogenous system of linear equations with more equations than variables has no solution for some choice of the constant terms $c$.
  \end{itemize}
  %
  \item[Matrix Of a Linear Map\label{itm:D6_matrix}]  $T\in\mathcal{L}(V,W)$ is given by \begin{align*}
  \mathcal{M}(T, (v_1,\cdots, v_n), (w_1, \cdots, w_m)) &=\begin{bmatrix}
  Tv_1 &| \cdots |& Tv_k & |\cdots |& Tv_n 
  \end{bmatrix} 
  \end{align*} by expanding every term of a base of $V$ as linear combinations of a base in $W$. Let us expand both sides of $w = Tv$:
  \begin{align*}
  \sum_i b_i w_i &= T\left(\sum_j c_jv_j\right) = \sum_j c_j T v_j = \sum_j c_j \left(\sum_i a_{ij} w_i\right) = \sum_i \left( \sum_j  a_{ij} c_j\right) w_i 
  \end{align*}
  Therefore the coefficients of every expansion are related by $b_i = \sum_j  a_{ij} c_j$, which can be expressed as:
  \begin{align*}
  \begin{bmatrix}
  b_1 \\
  \vdots \\
  b_i \\
  \vdots \\
  b_m
  \end{bmatrix}
  &= \begin{bmatrix}
  a_{11} &       &a_{1j} &        & a_{1n} \\
         &       &\vdots &        &        \\
  \vdots &\cdots &a_{ij} & \cdots &\vdots  \\
         &       &\vdots &        &        \\
  a_{m1} &       &a_{mj} &        &a_{mn}  \\
  \end{bmatrix}
  \begin{bmatrix}
  c_1 \\
  \vdots \\
  c_j \\
  \vdots \\
  c_n
  \end{bmatrix}
  \end{align*} where
  \begin{itemize}
  \item $c_i$ are the coefficients of the linear combination of $v$ on a base $\{v_j\}$ in $V$.
  \item $b_i$ are the coefficients of the linear combination of $w$ on a base $\{w_i\}$ in $W$.
  \item $a_{ij}$ coefficients of the linear combination of every vector in the base $v_j$ on the base $\{w_i\}$ in $W$.
  \end{itemize}
  Properties:
  \begin{itemize}
  \item $\mathcal{M}(T+S) = \mathcal{M}(T) + \mathcal{M}(S)$
  \item $\mathcal{M}(cT) = c\mathcal{M}(T)$
  \item $\mathcal{M}(PT) = \mathcal{M}(P)\mathcal{M}(T)$
  \end{itemize}
  whenever $c\in\mathbf{F}$, $T,S\in\mathcal{L}(U,V)$ and  $P\in\mathcal{M}(V,W)$
  %
  \item $\mathrm{Mat}(m,n,\mathbf{F})$ is the set of all $m\times n$ matrices with entries in $\mathbf{F}$. It is a vector space over $\mathbf{F}$
  \item \emph{Matrix of a vector} $v\in V$ is a $n\times 1$ matrix.
  \item[P14] $T\in\mathcal{L}(V,W)$ and $(v_1,\cdots,v_n)$ a base of $V$ and $(w_1,\cdots,w_m)$ a base of $W$. Then
  \begin{align*}
  \mathcal{M}(Tv) =\mathcal{M}(T)\mathcal{M}(v) 
  \end{align*}
  %
  \item[Invertibility\label{itm:D3_invertible_operator}] $T\in\mathcal{L}(V,W)$ \textbf{invertible} if $\exists S\in\mathcal{L}(W,V)$ and $ST=I_V$ (identity map on $V$) and $TS=I_U$ (identity map on $W$) 
  \begin{align*}
  I \circlearrowright V \xrightarrow{T}\xleftarrow{S} W \circlearrowleft I
  \end{align*}
  \item If $T$ invertible then it has a unique inverse denoted $T^{-1}$ 
  \item[P17] A linear map is invertible $\iff$ it is injective and surjective.
  
  \item[Isomorphic] Two vectors spaces are called isomorphic if there is an invertible linear map from one vector space onto the other one. As abstract vectors spaces, two isomorphic spaces have the same properties.
  
  \item[T18] Two finite dimensional vector spaces are isomorphic $\iff$ they have the same dimension
  
  \item[P19]
  
  \item[P20\label{itm:P3_20}] If $V$ and $W$ are finite dimensional vector spaces, then
  \begin{itemize}
  \item $\mathcal{L}(V,W)$ is a finite dimensional vector space
  \item $\dim{\mathcal{L}(V,W)} = \dim{V} \times \dim{W}$
  \end{itemize}
  
  \item[T21\label{itm:T3_21}] $V$ finite dimensional and $T\in\mathcal{L}(V)$ then
  \begin{align*}
  T \text{ invertible} \iff T \text{ injective} \iff T \text{ surjective} 
  \end{align*}
  
  
  \end{description}
  
  \hilight{TODO}
  
  %------------------------------------------------------------------------ 
  \subsection*{Exercises}
  \setcounter{paragraph}{0}
  
  \exo{} If $T \in \mathcal{L}(V,V)$ then by T3.4
  \begin{align*}
  \dim{V} = \underbrace{\dim{\nullspace{T}}}_{\leq 1} + \underbrace{\dim{\rangespace{T}}}_{\leq 1} = 1
  \end{align*}
  since $\nullspace{T}$ and $\rangespace{T}$ are subspaces of $V$ (source and destination domains are $V$) its dimensions are at most 1 (3.3 and 2.15). Two cases are possible
  \begin{itemize}
  \item $\dim{\rangespace{T}}=0$ and $\dim{\nullspace{T}}=1$ implies that $\rangespace{T}=\{0\}$ (subspace of dimension zero) and $\nullspace{T}=V$ (subspace with same dimension of containing vector space 2.12).
  \item $\dim{\rangespace{T}}=1$ and $\dim{\nullspace{T}}=0$ means that $\rangespace{T}=V$ and $\nullspace{T}=\{0\}$ 
  \end{itemize}
  
  $\rangespace{T}=V$ then $Tv = w$ can be expanded in a bases of $V$. Since $\dim{V}=1$ any vector $v\in V$ can be a base. Therefore $Tv = a v$ where $a\in \mathbf{F}$. Notice that if $a=0$, the first case is also included.
  
  %------------------------------------------------------------------------ 
  \exo{}
  \hilight{??}
  
  %------------------------------------------------------------------------ 
  \exo{} Let $S\in\mathcal{U,W}$ and $T\in\mathcal{V,W}$. Bv P2.13 we can extend subspace $U$ with another subspace $R$ of $V$ such that $V=U\oplus R$. This means that $\forall v\in V$ $!\exists u\in U$ and $r\in W$ such that $v=u+r$ and $Tv=T(u+r)=Tu+Tr$. Taking $Tv = Su +Tr$ we can conclude that if $u=v$ hence $Tu=Su$. 
  %------------------------------------------------------------------------ 
  \exo{} By T2.13 we can extend $\nullspace{T}$ subspace of $V$ (P3.1) to $V$ as $V = \nullspace{T} \oplus W$. Hence
  \begin{align*}
  \dimension{V} \stackrel{\text{T2.18}+\text{P1.6}}{=} \dimension{\nullspace{T}} + \dimension{W} \stackrel{\text{T3.4}}{=} \dimension{\nullspace{T}} + \dimension{\rangespace{T}}
  \end{align*} 
  but the $\rangespace{T}\subseteq \mathbf{F}$ (P3.3) since $T\in\mathcal{L}(V,\mathbf{F})$. Finally $\dimension{\rangespace{T}}\leq\dimension{F}=1$ (P2.15) which means, considering the equation above, that $\dimension{W}\leq 1$ and hence can be spanned by a single-vector base $W=\{au \mid a\in\mathrm{F}\}$\qed
  
  %------------------------------------------------------------------------ 
  \exo{} $(v_1,\cdots,v_n)$ l.i. in $V$ implies $0=\sum a_iv_i$ and $a_i=0\,\forall i$. $T$ injective $\stackrel{P3.2}{\iff} \nullspace{T}=\{0\}$ then $Tv_i\neq 0$ and $Tv_i\neq Tv_j$. Hence 
  \begin{align*}
  T0 = 0 = a_1 Tv_1 + \cdots + a_n Tv_n
  \end{align*}
  where all $a_i=0$, which means that $(Tv_1,\cdots,Tv_n)$ are l.i. \qed
  %------------------------------------------------------------------------ 
  \exo{} Let us assume that $S_1S_2\cdots S_nx = S_1S_2\cdots S_ny$. 
  Expanding both sides as $S_1S_2\cdots S_nx = S_1(S_2( \cdots S_n(x) ))$ lead to $S_1 x_1 = S_1 x_2$ but since $S_1$ is injective this implies tjat $x_1 = y_1$ or analogously $S_2( \cdots S_n(x)) = S_2( \cdots S_n(y))$. Repeating the same reasoning, $x=y$, which demonstrate that the product $S_1S_2\cdots S_n$ is injective.
  
  %------------------------------------------------------------------------ 
  \exo{} If $(v_1,\cdots,v_n)$ spans $V$ then $\forall v\in V$ $v=\sum a_i v_i$ with $a_i\in\mathbf{F}$. $T\in\mathcal{L}(V,W$ is surjective, and so it reaches any vector in $W$. Using the expression above, $\forall w\in W$ can be written as as linear superpositionc
  \begin{align*}
  w = Tv = \sum a_i Tv_i
  \end{align*}
  or in other words, the list $(Tv_1,\cdots, Tv_n)$ spans $W$.
  %------------------------------------------------------------------------ 
  \exo{}
  \hilight{??}
  %------------------------------------------------------------------------ 
  \exo{} Let  $T\in\mathcal{L}(\mathbf{F}^4, \mathbf{F}^2)$. $\nullspace{T}$ can be span with two linearly independent vectors as
  \begin{align*}
  x = a_1 (5,1,0,0) + a_2 (0,0,7,1)
  \end{align*}
  , therefore $\dimension{\nullspace T} = 2$. Using the identity from T4: $\dimension{\mathbf{F}^4} = 4 = 2 + \dimension{\rangespace{T}} \implies \dimension{\rangespace{T}} = \dimension{\mathbf{F}^2} = 2$ therefore $T$ is surjective. 
  %------------------------------------------------------------------------ 
  \exo{} This nullspace 
  \begin{align*}
  x =\beta (3,1,0,0,0) + \gamma (0, 0, 1,1,1)
  \end{align*}
  has dimension two which means (using T4) that $\dimension{\rangespace{T}} = 3$ but it cannot be since $\dimension{\mathbf{F}^2}=2$ and $\rangespace{T}\subseteq \mathbf{F}^2$.
  %------------------------------------------------------------------------ 
  \exo{} $\dimension{T} = \dimension{\nullspace{T}} + \dimension{\rangespace{T}} < \infty$
  %------------------------------------------------------------------------ 
  \exo{}
  \hilight{TODO - a limpio}
  %------------------------------------------------------------------------ 
  \exo{} $T\in\mathcal{L}(V,W)$ surjective $\iff$ $\dimension{V}\geq \dimension{W}$ (big $V$ to small $W$)
  \begin{itemize}
  \item[$\Rightarrow$] If $T$ surjective means $\rangespace{T}=W$ then by T3.4 $\dimension{V} = \underbrace{\dimension{\nullspace{T}}}_{\geq 0} + \dimension{W}$ therefore $\dimension{V}\geq \dimension{W}$. Another way to demonstrate that is proving the negation in the other direction. If $\dimension{W} \not{\leq} \dimension{V}$, i.e.  $\dimension{W} > \dimension{V}$, implies (by C3.6) that $T$ cannot be surjective
  
  \item[$\Leftarrow$]  \hilight{??}
  \end{itemize}\qed
  %------------------------------------------------------------------------ 
  \exo{} $T\in\mathcal{L}(V,W)$ and $\nullspace{T}=U \iff \dimension{U} \geq \dimension{V} - \dimension{W}$
  \begin{itemize}
  \item[$\Rightarrow$] $\dimension{V}=\dimension{\nullspace{T}} + \dimension{\rangespace{T}}\leq \dimension{\nullspace{T}} + \dimension{W}$ provided that $\rangespace{T}\subseteq W$. Hence $\dimension{U} \geq \dimension{V} - \dimension{W}$. 
  \item[$\Leftarrow$] \hilight{??}
  \end{itemize}
  
  %------------------------------------------------------------------------ 
  \exo{}
  
  \hilight{TODO}
  %%%%%%%%%%%%%%%%%%%%%%%%%%%%%%%%%%%%%%%%%%%%%%%%%%%%%%%%%%%%%%%
  \newpage
  \section{Polynomials}
  \subsection*{Summary}
  \hilight{TODO}
  \subsection*{Exercises}
  \setcounter{paragraph}{0}
  \hilight{TODO}
%%%%%%%%%%%%%%%%%%%%%%%%%%%%%%%%%%%%%%%%%%%%%%%%%%%%%%%%%%%%%%%
\newpage
\selectlanguage{english}
\small
\bibliography{refs}
\bibliographystyle{plain}

\end{document}
