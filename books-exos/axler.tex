% -*- Mode: LaTeX; tab-width: 4 -*-
%
% Work on Sheldon Axler 'Linear Algebra Done Right'.
% Pedro Crespo Valero - 2015, 2021

\documentclass[11pt,notitlepage,oneside]{article}
\author{Pedro Crespo-Valero}

\usepackage[margin=1.5cm]{geometry} % to tell the pdf about our paper size
\usepackage[pdfpagelabels,pdfpagemode=None,backref=page,pagebackref=true,
colorlinks=true,bookmarks=true,bookmarksnumbered=true]{hyperref}
\usepackage{nameref}
\usepackage[latin1]{inputenc} % accents, REQUIRED for non english languages
\usepackage[english]{babel}
\usepackage{graphicx} % anything beyond pure text requires this
\usepackage[centertags,sumlimits,nointlimits,reqno]{amsmath}
\usepackage{amssymb}
\usepackage{amsbsy}
\usepackage{amsxtra}
\usepackage{amsthm}
\usepackage{mathtools}
\usepackage[nottoc,notbib]{tocbibind} % to put the bibliography in the toc
\usepackage[square,comma,numbers,sort&compress]{natbib} % many options in bibl.
\usepackage{tikz}
\usetikzlibrary{calc}
\usetikzlibrary{shapes,backgrounds}
%\usepackage{color}
%\usepackage{enumitem}
\usepackage{pifont}% http://ctan.org/pkg/pifont
\usepackage[printonlyused,withpage]{acronym}

%% NEW COMMANDS ------
\newcommand{\cmark}{\ding{51}}%
\newcommand{\xmark}{\ding{55}}%
\newcommand{\hilight}[1]{\colorbox{yellow}{#1}}
\newcommand{\pfrac}[2]{\frac{\partial #1}{\partial #2}}
\newcommand{\diff}{\mathrm{d}}
\newcommand{\firstsaid}[1]{\textbf{#1}}
\newcommand{\relphantom}[1]{\mathrel{\phantom{#1}}}
\newcommand{\wedgeseq}{\wedge\ldots\wedge}
\newcommand{\hodge}{\star}
\newcommand{\laplacian}{\Delta}
\DeclareMathOperator{\rot}{curl}
\DeclareMathOperator{\divg}{div}
\DeclareMathOperator{\dimension}{dim}
\DeclareMathOperator{\grad}{grad}
\DeclareMathOperator{\trace}{trace}
\DeclareMathOperator{\conv}{conv}

\renewcommand{\theparagraph}{\arabic{section}.\arabic{paragraph}}
\newcommand{\exref}[1]{Ex.\ref{#1}}
\newcommand{\figref}[1]{Fig.\ref{#1}}
\newcommand{\pref}[1]{P\ref{#1}}
\newcommand{\tbc}{\ensuremath{\ \oplus}}
\newcommand{\exo}[1]{%
\addtocontents{toc}{\protect\setcounter{tocdepth}{2}}%
\paragraph{#1}}
\newcommand{\nullspace}[1]{\mathrm{null}\,{#1}}
\newcommand{\rangespace}[1]{\mathrm{range}\,{#1}}
\newcommand{\inner}[1]{\langle {#1}\rangle}
\newcommand{\norm}[1]{\|#1\|}
\newcommand{\jota}{\mathrm{j}}

% From DLL

% A4 reasonable margins (ie not wasting margins).

\usepackage{calc} % dimensions, tipically
\usepackage{subfigure}
\usepackage{fancyhdr} % header-footer
\usepackage{fancybox} % boxes

\setlength{\paperwidth}{160mm}
\setlength{\paperheight}{250mm}
\setlength{\hoffset}{-1in} % Compensate the driver margins
\setlength{\voffset}{-1in} % Compensate the driver margins
\setlength{\oddsidemargin}{5mm} % original 20 25 % then 15 45
\setlength{\evensidemargin}{5mm}
%\setlength{\evensidemargin}{.382\paperwidth-\oddsidemargin}
\newlength{\saveoddsidemargin}\setlength{\saveoddsidemargin}{\oddsidemargin}
\newlength{\saveevensidemargin}\setlength{\saveevensidemargin}{\evensidemargin}
% original 20 25 % then 15 40
\newlength{\headoddsidemargin}\setlength{\headoddsidemargin}{\oddsidemargin}
\newlength{\headevensidemargin}\setlength{\headevensidemargin}{\oddsidemargin}
\setlength{\textwidth}{\paperwidth-\oddsidemargin-\evensidemargin}
\setlength{\headwidth}{\paperwidth-\headoddsidemargin-\headevensidemargin}
\setlength{\headheight}{15pt} % 15 pt
\setlength{\headsep}{10pt} % 25 pt
\setlength{\topmargin}{5mm}
\setlength{\footskip}{0mm}
\newlength{\bottommargin} % 20mm
\setlength{\bottommargin}{5mm}
\setlength{\textheight}{\paperheight-\topmargin-\headheight-\headsep-\footskip-\bottommargin}
\renewcommand{\baselinestretch}{1}
\flushbottom % same text height for all the pages


%: ->  foot

% These commented out because caption2 deprecated, would need replacements in caption.
%\setlength{\captionmargin}{10pt}
%\setlength{\abovecaptionskip}{10pt}
%\setlength{\belowcaptionskip}{0pt}
%\renewcommand{\captionfont}{\sffamily}
%\makeatletter
%\newcommand\figcaption{%
%	\renewcommand{\figurename}{Fig.}\def\@captype{figure}\caption}
%\newcommand\tabcaption{\def\@captype{table}\caption}
%\makeatother

%: -> figures

\renewcommand{\subfigtopskip}{10pt}
\renewcommand{\subfigcapskip}{5pt}
\renewcommand{\subfigbottomskip}{10pt}
\newcommand{\goodgap}{%
	\hspace{\subfigtopskip}%
	\hspace{\subfigbottomskip}}

%: -> line spacing

\sloppy % imprescindible...
\frenchspacing
\setlength{\parindent}{2em}
\setlength{\parskip}{0ex}

%: -> to adjust properly the figure

\renewcommand{\floatpagefraction}{.8}
\newcommand{\topnumber}{2}
\renewcommand{\topfraction}{.8}
\renewcommand{\bottomfraction}{.8}
\renewcommand{\textfraction}{0.0}

%: -> fancy pagestyle

\pagestyle{fancy}
\renewcommand{\headrulewidth}{0pt}
\fancyhead[RO]{\flushright\thepage}
\fancyhead[LO]{\rightmark}
\renewcommand{\footrulewidth}{0.0pt}
\renewcommand{\sectionmark}[1]{%
	\markright{\thesection.\ #1}{}}
\fancyhf{}
\fancyhead[L]{\thepage}
\fancyhead[RO]{\thepage}
\fancyhead[LO]{\rightmark}
\fancyhead[R]{\sc\leftmark}
\fancypagestyle{plain}{\fancyhf{}}

\usepackage{parskip}
\linespread{0.95}
\interfootnotelinepenalty=10000

\selectlanguage{english}
\setcounter{secnumdepth}{10}
\setcounter{tocdepth}{1}


% http://tex.stackexchange.com/questions/1230/reference-name-of-description-list-item-in-latex
\makeatletter
\let\orgdescriptionlabel\descriptionlabel
\renewcommand*{\descriptionlabel}[1]{%
  \let\orglabel\label%
  \let\label\@gobble%
  \phantomsection%
  \edef\@currentlabel{#1}%
  \edef\@currentlabelname{#1}%
  \let\label\orglabel%
  \orgdescriptionlabel{#1}%
}
\makeatother


% MAIN DOCUMENT ----------------------------------------------------------------
\begin{document}
\title{From \emph{Linear Algebra Done Right}~\cite{Axler1997}}
\date{2015/9}
\date{2021/10}
\maketitle

\tableofcontents

\newpage
\setcounter{section}{0}


%%%%%%%%%%%%%%%%%%%%%%%%%%%%%%%%%%%%%%%%%%%%%%%%%%%%%%%%%%%%%%%
\input{axler.01.tex}


%%%%%%%%%%%%%%%%%%%%%%%%%%%%%%%%%%%%%%%%%%%%%%%%%%%%%%%%%%%%%%%
\newpage
\section{Finite-Dimensional Vector Spaces}
\subsection*{Summary}
\begin{itemize}
\item A list of vectors $(v_1,\cdots,v_n)$ is \textbf{linearly independent} in a vector space $V$ $\iff 0 = \sum a_i v_i$
and the only choice is $a_1=\cdots=a_n=0$.
%
\item \textbf{Basis} of vector space $V$ is a list of vectors in $V$ that is \emph{linearly indepedent} and \emph{spans} $V$.
\begin{itemize}
\item $(e_1, \cdots, e_n)$ where $e_i=(0,\cdots, \underbrace{1}_{i},\cdots,0)\in\mathbf{F}^n$ is the standard basis of $\mathbf{F}^n$. 
\end{itemize}
\item[P8] A list $(v_1,\cdots,v_n)$ of vectors in $V$ is a basis of $V$ $\iff$ $\forall v\in V$ can be written uniquely as a linear combination of these vectors $v=\sum a_i v_i$.
\item[T10/12] Every spanning/linearly independent  list of vectors in a finite-dimensional vector space can be \emph{reduced}/\emph{extended} to a basis of the vector space.
\item[C11] Every finite dimensional vector space \emph{has} a basis.
\item[T13] $V$ finite dimensional and $U$ subspace of $V$ $\implies$ there is a subspace $W$ of $V$ such that $V=U\oplus W$.
%
\item The \textbf{dimension} of a finite-dimensional vectors spaces is the \emph{length} of any basis of the vector space.
\item[P15] $V$ finite dimensional and $U$ subspace of $V$ $\implies$ $\dimension{U}\leq\dimension{V}$.
%
\item[P16/17]Every spanning/linearly independent list of vectors in $V$ with the right length ($=\dimension{V}$) is a basis of $V$ 
%
\item[T18] If $U_1,U_2$ subspaces of a finite-dimensional vector space then 
\begin{align*}
\dimension(U_1+U_2)=\dimension(U_1)+\dimension(U_2) - \dimension(U_1\cap U_2)
\end{align*}
\item[P19] $V$ finite dimensional vector space and $U_1,\cdots,U_m$ subspaces of $V$ such that $V=\sum U_i$ and $\dimension{V}=\sum_i\dimension{U_i}$ then $V=\bigoplus U_i$
\end{itemize}
\hilight{TODO}
%------------------------------------------------------------------------ 
\subsection*{Exercises}
\hilight{TODO a limpio}


%%%%%%%%%%%%%%%%%%%%%%%%%%%%%%%%%%%%%%%%%%%%%%%%%%%%%%%%%%%%%%%
\newpage
\section{Linear Maps}
%------------------------------------------------------------------------ 
\subsection*{Summary}

%%%%%%%%%%%%%%%%%%%%%%%%%%%%%%%%%%%%%%%%%%%%%%%%%%%%%%%%%%%%%%%
\newpage
\selectlanguage{english}
\small
\bibliography{refs}
\bibliographystyle{plain}

\end{document}
