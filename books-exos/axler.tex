% -*- Mode: LaTeX; tab-width: 4 -*-
%
% Work on Sheldon Axler 'Linear Algebra Done Right'.
% Pedro Crespo Valero - 2015, 2021

\documentclass[11pt,notitlepage,oneside]{article}
\author{Pedro Crespo-Valero}

\usepackage[margin=1.5cm]{geometry} % to tell the pdf about our paper size
\usepackage[pdfpagelabels,pdfpagemode=None,backref=page,pagebackref=true,
colorlinks=true,bookmarks=true,bookmarksnumbered=true]{hyperref}
\usepackage{nameref}
\usepackage[latin1]{inputenc} % accents, REQUIRED for non english languages
\usepackage[english]{babel}
\usepackage{graphicx} % anything beyond pure text requires this
\usepackage[centertags,sumlimits,nointlimits,reqno]{amsmath}
\usepackage{amssymb}
\usepackage{amsbsy}
\usepackage{amsxtra}
\usepackage{amsthm}
\usepackage{mathtools}
\usepackage[nottoc,notbib]{tocbibind} % to put the bibliography in the toc
\usepackage[square,comma,numbers,sort&compress]{natbib} % many options in bibl.
\usepackage{tikz}
\usetikzlibrary{calc}
\usetikzlibrary{shapes,backgrounds}
%\usepackage{color}
%\usepackage{enumitem}
\usepackage{pifont}% http://ctan.org/pkg/pifont
\usepackage[printonlyused,withpage]{acronym}

%% NEW COMMANDS ------
\newcommand{\cmark}{\ding{51}}%
\newcommand{\xmark}{\ding{55}}%
\newcommand{\hilight}[1]{\colorbox{yellow}{#1}}
\newcommand{\pfrac}[2]{\frac{\partial #1}{\partial #2}}
\newcommand{\diff}{\mathrm{d}}
\newcommand{\firstsaid}[1]{\textbf{#1}}
\newcommand{\relphantom}[1]{\mathrel{\phantom{#1}}}
\newcommand{\wedgeseq}{\wedge\ldots\wedge}
\newcommand{\hodge}{\star}
\newcommand{\laplacian}{\Delta}
\DeclareMathOperator{\rot}{curl}
\DeclareMathOperator{\divg}{div}
\DeclareMathOperator{\dimension}{dim}
\DeclareMathOperator{\grad}{grad}
\DeclareMathOperator{\trace}{trace}
\DeclareMathOperator{\conv}{conv}

\renewcommand{\theparagraph}{\arabic{section}.\arabic{paragraph}}
\newcommand{\exref}[1]{Ex.\ref{#1}}
\newcommand{\figref}[1]{Fig.\ref{#1}}
\newcommand{\pref}[1]{P\ref{#1}}
\newcommand{\tbc}{\ensuremath{\ \oplus}}
\newcommand{\exo}[1]{%
\addtocontents{toc}{\protect\setcounter{tocdepth}{2}}%
\paragraph{#1}}
\newcommand{\nullspace}[1]{\mathrm{null}\,{#1}}
\newcommand{\rangespace}[1]{\mathrm{range}\,{#1}}
\newcommand{\inner}[1]{\langle {#1}\rangle}
\newcommand{\norm}[1]{\|#1\|}
\newcommand{\jota}{\mathrm{j}}

% From DLL

% A4 reasonable margins (ie not wasting margins).
% Originaly copied and modified from https://github.com/lloda

\usepackage{calc} % dimensions, tipically
\usepackage{subfigure}
\usepackage{fancyhdr} % header-footer
\usepackage{fancybox} % boxes

\setlength{\paperwidth}{160mm}
\setlength{\paperheight}{250mm}
\setlength{\hoffset}{-1in} % Compensate the driver margins
\setlength{\voffset}{-1in} % Compensate the driver margins
\setlength{\oddsidemargin}{5mm} % original 20 25 % then 15 45
\setlength{\evensidemargin}{5mm}
%\setlength{\evensidemargin}{.382\paperwidth-\oddsidemargin}
\newlength{\saveoddsidemargin}\setlength{\saveoddsidemargin}{\oddsidemargin}
\newlength{\saveevensidemargin}\setlength{\saveevensidemargin}{\evensidemargin}
% original 20 25 % then 15 40
\newlength{\headoddsidemargin}\setlength{\headoddsidemargin}{\oddsidemargin}
\newlength{\headevensidemargin}\setlength{\headevensidemargin}{\oddsidemargin}
\setlength{\textwidth}{\paperwidth-\oddsidemargin-\evensidemargin}
\setlength{\headwidth}{\paperwidth-\headoddsidemargin-\headevensidemargin}
\setlength{\headheight}{15pt} % 15 pt
\setlength{\headsep}{10pt} % 25 pt
\setlength{\topmargin}{5mm}
\setlength{\footskip}{0mm}
\newlength{\bottommargin} % 20mm
\setlength{\bottommargin}{5mm}
\setlength{\textheight}{\paperheight-\topmargin-\headheight-\headsep-\footskip-\bottommargin}
\renewcommand{\baselinestretch}{1}
\flushbottom % same text height for all the pages


%: ->  foot

% These commented out because caption2 deprecated, would need replacements in caption.
%\setlength{\captionmargin}{10pt}
%\setlength{\abovecaptionskip}{10pt}
%\setlength{\belowcaptionskip}{0pt}
%\renewcommand{\captionfont}{\sffamily}
%\makeatletter
%\newcommand\figcaption{%
%	\renewcommand{\figurename}{Fig.}\def\@captype{figure}\caption}
%\newcommand\tabcaption{\def\@captype{table}\caption}
%\makeatother

%: -> figures

\renewcommand{\subfigtopskip}{10pt}
\renewcommand{\subfigcapskip}{5pt}
\renewcommand{\subfigbottomskip}{10pt}
\newcommand{\goodgap}{%
	\hspace{\subfigtopskip}%
	\hspace{\subfigbottomskip}}

%: -> line spacing

\sloppy % imprescindible...
\frenchspacing
\setlength{\parindent}{2em}
\setlength{\parskip}{0ex}

%: -> to adjust properly the figure

\renewcommand{\floatpagefraction}{.8}
\newcommand{\topnumber}{2}
\renewcommand{\topfraction}{.8}
\renewcommand{\bottomfraction}{.8}
\renewcommand{\textfraction}{0.0}

%: -> fancy pagestyle

\pagestyle{fancy}
\renewcommand{\headrulewidth}{0pt}
\fancyhead[RO]{\flushright\thepage}
\fancyhead[LO]{\rightmark}
\renewcommand{\footrulewidth}{0.0pt}
\renewcommand{\sectionmark}[1]{%
	\markright{\thesection.\ #1}{}}
\fancyhf{}
\fancyhead[L]{\thepage}
\fancyhead[RO]{\thepage}
\fancyhead[LO]{\rightmark}
\fancyhead[R]{\sc\leftmark}
\fancypagestyle{plain}{\fancyhf{}}

\usepackage{parskip}
\linespread{0.95}
\interfootnotelinepenalty=10000

\selectlanguage{english}
\setcounter{secnumdepth}{10}
\setcounter{tocdepth}{1}


% http://tex.stackexchange.com/questions/1230/reference-name-of-description-list-item-in-latex
\makeatletter
\let\orgdescriptionlabel\descriptionlabel
\renewcommand*{\descriptionlabel}[1]{%
  \let\orglabel\label%
  \let\label\@gobble%
  \phantomsection%
  \edef\@currentlabel{#1}%
  \edef\@currentlabelname{#1}%
  \let\label\orglabel%
  \orgdescriptionlabel{#1}%
}
\makeatother


% MAIN DOCUMENT ----------------------------------------------------------------
\begin{document}
\title{From \emph{Linear Algebra Done Right}~\cite{Axler1997}}
\date{2015/9}
\date{2021/10}
\maketitle

\tableofcontents

\newpage
\setcounter{section}{0}

%%%%%%%%%%%%%%%%%%%%%%%%%%%%%%%%%%%%%%%%%%%%%%%%%%%%%%%%%%%%%%%
\section{Vector Spaces}
\subsection*{Summary}
% Summary -----------------------------------------------------
\begin{itemize}
\item \textbf{scalars} $\mathbb{F}$ (i.e. fields $\mathbb{R}$ or $\mathbb{C}$)
\begin{itemize}
\item commutativity +, $\cdot$
\item associativity +, $\cdot$
\item identity +0, $\cdot 1$
\item \emph{additive} inverse: $\exists!w \mid z + w = 0$, denoted $-z$
\item \emph{multiplicative} inverse: $\exists!w$ and $z\neq0$ then $z\cdot w = 1$, denoted $1/w$
\item distributive $a(b+c) = ab+ac$
\end{itemize}
%---
\item A \textbf{vector space} is a set $V$ along with an addition on $V$ and a scalar multiplication on $V$ that satisfies:
\begin{itemize}
\item commutativity +
\item associativity +
\item additive identity: $0\in V$ such that $v+0=v$ $\forall v\in V$
\item additive inverse: $\forall v\in V,\; \exists!w\in V$ such that $v+w=0$, denoted $-v$
\item multiplicative identify: $1v=v\;\forall v\in V$ and $1\in\mathbb{F}$
\item distributive: $a(u+v)=au+av$ and $(a+b)u=au+bu$ $\forall a,b\in\mathbb{F}$ and $\forall u,v \in V$.
\end{itemize}
 
\item[P1:\label{it:P1_1}] A \emph{vector space} has a \emph{unique} additive identity $0$
\item[P2:\label{it:P1_2}] Every \emph{element} of a vector space has a \emph{unique} additive inverse $v\leftarrow -v$
\item[P3:] $0v=0$ $\forall v\in V$
\item[P4:] $a0=0$ $\forall a\in \mathbb{F}$
\item[P5:] $(-1)v = -v$ $\forall v\in V$, i.e. the additive inverse

\item \textbf{subspace}: $U\subset V$ is a subspace of $V$ if $U$ is also a vector space, that is
\begin{itemize}
\item has additive identity: $0\in U$
\item closed under addition $\forall u,v \in V$: $u+v \in U$ 
\item closed under scalar multiplication $\forall \lambda\in \mathbb{F}$ and $\forall u \in V$: $\lambda u \in U$  
\item remaining conditions for a vector space are automatically satisfied since $U\subset V$.
\end{itemize} 

\item \textbf{sum of subspaces}: $U_1+\cdots U_n = \sum U_i =\{\sum u_i\mid u_i\in U_i\}$ and $U_i\subset V$ subspaces of $V$. $\sum U_i$ is
\begin{itemize}
\item a subspace of $V$.
\item the smallest subspace of $V$ that contains all $U_i$.
\end{itemize}

\item $V$ is a \textbf{direct sum of subspaces} $U_i$, written $V=U_1\oplus\cdots\oplus U_n$, if each element of $V$ can be written \emph{uniquely} as a sum $\sum u_i$, where each $u_i\in U_i$. 
Then $V=\bigoplus U_i \iff$
\begin{itemize}
\item $V = \sum U_i$
\item $0=\sum u_i$ and $u_i=0 \in U_i$.
\end{itemize}

\item[P6:\label{it:P1_6}] $U,W$ subspaces of $V$ and $V=U\oplus W \iff$
\begin{itemize}
\item $V=U+V$
\item $U\cap V=\{0\}$
\end{itemize}
but \emph{only} valid for \emph{pairs} of subspaces!

\end{itemize}

\subsection*{Exercises}
%------------------------------------------------------------------------ 
\exo{}
\begin{align*}
\frac{1}{z} = 1 \frac{1}{z} = (z^* \cdot \frac{1}{z^*}) \frac{1}{z} = \frac{z^*}{zz^*}=\frac{a-\mathrm{j}b}{a^2 + b^2} = c+\mathrm{j}d
\end{align*}
where we use first identity, then multiplicative inverse in $\mathbb{C}$, and finally by replacing $z=a+jb$, we use distributive property 
to get $zz^* = a^2 + b^2 $\qed

%------------------------------------------------------------------------ 
\exo{}
\begin{align*}
\sqrt[3]{1} = \exp{\left(\mathrm{j}\frac{2\pi k}{3}\right)} = \cos(2\pi k/3) +\mathrm{j}\sin(2\pi k/3) = \left\{ 1, \frac{-1+\sqrt{3}\mathrm{j}}{2}, \frac{-1-\sqrt{3}\mathrm{j}}{2} \right\} 
\end{align*}
for $k=0,1,2$.  \qed
%------------------------------------------------------------------------ 
\exo{} Applying P5 twice and the multiplicative identity
\begin{align*}
-(-v)= (-1)(-1)v = 1v = v  
\end{align*}\qed
%------------------------------------------------------------------------ 
\exo{} \textit{Reductio ad absurdum}: Let $av=0$ with $a\neq 0 \in \mathbb{F}$, $v\neq0 \in V$. Using the scalar's multiplicative inverse property, if $a\neq 0$ there exists a unique $b\in \mathbb{F}$ that inverses it $ab = ba = 1$. Then
\begin{align*}
av &= 0 & \\ 
bav &= b0 & \\
1v &= 0 & \text{mult. inverse \& conmutative + P3}\\ 
v &=0 & \text{mult. identity}
\end{align*}
which is a contradiction. 
\qed
%------------------------------------------------------------------------ 
\exo{}
\begin{enumerate}
\item[a)] Yes. $V=\{x\in\mathbb{F}^3\mid vx^T=0$ with $v=(1,2,3)\in \mathbb{F}^3 \}$. A plane through the origin is a subspace over $\mathbb{F}^3$ since $\forall x_1, x_2 \in V$
\begin{itemize}
\item Contains $0$, since $v0^T= 0$
\item Closed under $+$: $v(x_1^T + x_2^T) = vx_1^T + vx_2^T= 0+0 = 0$ (distributive )
\item Closed under scalar $\times$: $v \lambda x^T = \lambda (v x^T) = \lambda 0 = 0$  (commutative and P4)
\end{itemize}

\item[b)] No. $V=\{x\in\mathbb{F}^3\mid vx^T=b\}$ with  $v=(1,2,3)$ and $b=4$ is an affine set that does not contain $0$ since $v0\neq 4$. Therefore is \emph{not} a subspace. Can also be written as $v(x-x_0)^T = 0$ or $(x_1-4) + 2x_2 + 3x_3=0$.

\item[c)] No.  $V=\{x\in\mathbb{F}^3\mid x_1x_2x_3=0\}$ is the union of three planes $x_i=0$ (since either $x_1=0$ or $x_2=0$ or $x_3=0$ are solutions). 
This case includes $0$ but is not closed under sum because $\forall x,y\in V$ then $\Pi_i(x_i+y_i)-\Pi_i(x_i)-\Pi_i(y_i) \neq 0$. 
Each plane independently $x_i=0$  is a subspace but its union (careful, not it's \emph{sum}) in $V$ is \emph{not} a subspace.

\item[d)]Yes. $V=\{x\in\mathbb{F}^3\mid x_1=5x_3\}$ is a line through the origin and is a subspace of $\mathbb{F}^3$:
\begin{itemize}
\item $0\in V$
\item closed under $+$: $ x_1+y_1 = 5x_3+5y_3 = 5(x_3+y_3)$
\item closed under scalar $\times$: $\lambda x_1 = \lambda 5x_3$
\end{itemize}

\end{enumerate}
%------------------------------------------------------------------------ 
\exo{}
$V = \mathbb{I}^2 \in \mathbb{R}^2$ where every point is a pair of integers $x=(m,n)$. This case includes additive identity, is closed under sum since the sum of integers are integers, also under additive inverses since subtracting integers gives integers but \emph{not} closed under scalar multiplication since an integer times a real number is in general real.

%------------------------------------------------------------------------ 
\exo{}
Any line or plane not crossing the origin are closed under addition and scalar multiplication but do not contain the additive identity.

%------------------------------------------------------------------------ 
\exo{}$\cap U_i$ and $U_i$ subspaces of $V$. 
\begin{itemize}
\item $0 \in \cap U_i$ since $0\in U_i$
 \item $x,y \in \cap U_i \implies x,y \in U_i\; \forall i \implies x+y \in U_i\; \forall i \implies x+y\in \cap U_i$
  \item analogous scalar product
\end{itemize}
therefore $\cap U_i$ is a subspace of $V$.\qed

%------------------------------------------------------------------------ 
\exo{} Let $A,B$ subspaces of $V$
\begin{itemize}
\item[$\leftarrow$)]$A\subset B \subset V$ then $A\cup B = B$ which is already a subspace of $V$.
\item[$\rightarrow$)] $A\cup B$ is a subspace $V$ \hilight{... ??}
\end{itemize}

%------------------------------------------------------------------------ 
\exo{}$U+U =\{ u+v \mid \forall u,v \in U\}$ since $U$ is a subspace of $V$ then $u+v\in U$ therefore $U+U=U$.

%------------------------------------------------------------------------ 
\exo{} yes since the sum of subspaces is a subspace and all elements in the subspace satisfy commutativity and associativity. 
%------------------------------------------------------------------------ 
\exo{} \begin{itemize}
\item $\sum U_i$ is a subspace and any subspace, by \pref{it:P1_1}, has a unique additive identity. Therefore $\sum U_i + \sum 0 = \sum U_i$.
\item In \pref{it:P1_2}, every element in a vector space has  a unique additive inverse. Therefore $\sum U_i - \sum U_i = \{0\}$.
\end{itemize}
%------------------------------------------------------------------------ 
\exo{} No. For instance, since any line through the origin is a subspace of $\mathbf{R}^2$, then by \pref{it:P1_6}, we also know 
that $\mathbf{R}^2$ is a direct sum of any pair of these lines (subspaces) $U_i, W$. 
An example is
\begin{align*}
  U_1 &= \{(x,0) \in \mathbb{R}^2 :x \in \mathbb{R} \}  \\
  U_2 &= \{(x,x)\in \mathbb{R}^2 :x \in \mathbb{R}  \}  \\
  W   &= \{(0,y)\in \mathbb{R}^2 :y \in \mathbb{R}  \}  \\
\end{align*}
and $\mathbf{R}^2 = U_1 \oplus W = U_2 \oplus W $ .

%------------------------------------------------------------------------ 
\exo{} Let $W$ be a subspace of $\mathcal{P}(\mathbf{F})$ consisting of all polynomials as
\begin{align}
  q(z) = \sum\limits_{i\neq2,5} c_i z
\end{align}
then $(p+q)(z) = c_0 + c_1z + a z^2+ c_3 z^3 + c_4 z^4 + b z^5 + c_6 z^6 + ...$ and therefore $\mathcal{P}(\mathbf{F}) = U \oplus W$. 

%------------------------------------------------------------------------ 
\exo{} No. Counterexample: take three lines as subspaces of $V=\mathbf{R}^2$ such as $U_1=\{(x,0)\}, U_2=\{(x,x)\}, W= \{(0,x)\}$ with $x\in \mathbf{R}$. Since
$U_i\cap W = \{0\}$ then by \pref{it:P1_6}, $\mathbf{R}^2$ is direct sum of subspaces $U_1,W$ or $U_2,W$ but still $U_1\neq U_2$. 
%------------------------------------------------------------------------ 

%%%%%%%%%%%%%%%%%%%%%%%%%%%%%%%%%%%%%%%%%%%%%%%%%%%%%%%%%%%%%%%
\newpage
\section{Finite-Dimensional Vector Spaces}
\subsection*{Summary}

%%%%%%%%%%%%%%%%%%%%%%%%%%%%%%%%%%%%%%%%%%%%%%%%%%%%%%%%%%%%%%%
\newpage
\selectlanguage{english}
\small
\bibliography{refs}
\bibliographystyle{plain}

\end{document}
